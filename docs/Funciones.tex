% Created 2018-04-09 lun 18:32
% Intended LaTeX compiler: pdflatex
\documentclass[xcolor={usenames,svgnames,dvipsnames}]{beamer}
\usepackage[utf8]{inputenc}
\usepackage[T1]{fontenc}
\usepackage{graphicx}
\usepackage{grffile}
\usepackage{longtable}
\usepackage{wrapfig}
\usepackage{rotating}
\usepackage[normalem]{ulem}
\usepackage{amsmath}
\usepackage{textcomp}
\usepackage{amssymb}
\usepackage{capt-of}
\usepackage{hyperref}
\usepackage{color}
\usepackage{listings}
\usepackage[spanish]{babel}
\usecolortheme{rose}
\setbeamercolor{alerted text}{fg=Blue}
\setbeamerfont{alerted text}{series=\bfseries}
\setbeamercolor{block title}{bg=structure.fg!20!bg!50!bg}
\setbeamercolor{block body}{use=block title,bg=block title.bg}
\setbeamertemplate{navigation symbols}{}
\AtBeginSubsection[]{\begin{frame}[plain]\tableofcontents[currentsubsection,sectionstyle=show/shaded,subsectionstyle=show/shaded/hide]\end{frame}}
\lstset{keywordstyle=\color{blue}, commentstyle=\color{gray!90}, basicstyle=\ttfamily\small, columns=fullflexible, breaklines=true,linewidth=\textwidth, backgroundcolor=\color{gray!23}, basewidth={0.5em,0.4em}, literate={á}{{\'a}}1 {ñ}{{\~n}}1 {é}{{\'e}}1 {ó}{{\'o}}1 {º}{{\textordmasculine}}1, showstringspaces=false}
\usepackage{mathpazo}
\hypersetup{colorlinks=true, linkcolor=Blue, urlcolor=Blue}
\usepackage{fancyvrb}
\DefineVerbatimEnvironment{verbatim}{Verbatim}{fontsize=\tiny, formatcom = {\color{black!70}}}
\beamertemplatenavigationsymbolsempty
\setbeamertemplate{footline}[frame number]
\AtBeginSection[]{\begin{frame}[plain]\tableofcontents[currentsection,sectionstyle=show/shaded]\end{frame}}
\usetheme{Goettingen}
\usefonttheme{serif}
\author{Oscar Perpiñán Lamigueiro \\ \url{http://oscarperpinan.github.io}}
\date{}
\title{Funciones}
\hypersetup{
 pdfauthor={Oscar Perpiñán Lamigueiro \\ \url{http://oscarperpinan.github.io}},
 pdftitle={Funciones},
 pdfkeywords={},
 pdfsubject={},
 pdfcreator={Emacs 25.2.2 (Org mode 9.1.9)}, 
 pdflang={Spanish}}
\begin{document}

\maketitle


\section{Conceptos Básicos}
\label{sec:org4859b94}

\begin{frame}[label={sec:orgae0d9f2}]{Fuentes de información}
\begin{itemize}
\item \href{http://cran.r-project.org/doc/manuals/R-intro.html}{R introduction}
\item \href{http://cran.r-project.org/doc/manuals/R-lang.html}{R Language Definition}
\item \href{http://www.springer.com/gb/book/9780387759357}{Software for Data Analysis}
\end{itemize}
\end{frame}
\begin{frame}[fragile,label={sec:orgb9891dc}]{Componentes de una función}
 \begin{itemize}
\item Una función se define con \texttt{function}
\end{itemize}
\begin{center}
\texttt{name <- function(arg\_1, arg\_2, ...) expression}
\end{center}
\begin{itemize}
\item Está compuesta por:
\begin{itemize}
\item Nombre de la función (\texttt{name})
\item Argumentos (\texttt{arg\_1}, \texttt{arg\_2}, \texttt{...})
\item Cuerpo (\texttt{expression}): emplea los argumentos para generar un resultado
\end{itemize}
\end{itemize}
\end{frame}
\begin{frame}[fragile,label={sec:org99976c6}]{Mi primera función}
 \begin{itemize}
\item Definición
\end{itemize}
\lstset{language=r,label= ,caption= ,captionpos=b,numbers=none}
\begin{lstlisting}
myFun <- function(x, y)
{
    x + y
}
\end{lstlisting}

\begin{itemize}
\item Argumentos
\end{itemize}
\lstset{language=r,label= ,caption= ,captionpos=b,numbers=none}
\begin{lstlisting}
formals(myFun)
\end{lstlisting}

\begin{verbatim}
$x


$y
\end{verbatim}

\begin{itemize}
\item Cuerpo
\end{itemize}
\lstset{language=r,label= ,caption= ,captionpos=b,numbers=none}
\begin{lstlisting}
body(myFun)
\end{lstlisting}

\begin{verbatim}
{
    x + y
}
\end{verbatim}
\end{frame}

\begin{frame}[fragile,label={sec:org29d6f2f}]{Mi primera función}
 \lstset{language=r,label= ,caption= ,captionpos=b,numbers=none}
\begin{lstlisting}
myFun(1, 2)
\end{lstlisting}

\begin{verbatim}
[1] 3
\end{verbatim}

\lstset{language=r,label= ,caption= ,captionpos=b,numbers=none}
\begin{lstlisting}
myFun(1:10, 21:30)
\end{lstlisting}

\begin{verbatim}
[1] 22 24 26 28 30 32 34 36 38 40
\end{verbatim}

\lstset{language=r,label= ,caption= ,captionpos=b,numbers=none}
\begin{lstlisting}
myFun(1:10, 3)
\end{lstlisting}

\begin{verbatim}
[1]  4  5  6  7  8  9 10 11 12 13
\end{verbatim}
\end{frame}

\begin{frame}[fragile,label={sec:org19d267f}]{Argumentos: nombre y orden}
 Una función identifica sus argumentos por su nombre y por su orden (sin nombre)

\lstset{language=r,label= ,caption= ,captionpos=b,numbers=none}
\begin{lstlisting}
power <- function(x, exp)
{
    x^exp
}
\end{lstlisting}

\lstset{language=r,label= ,caption= ,captionpos=b,numbers=none}
\begin{lstlisting}
power(x=1:10, exp=2)
\end{lstlisting}

\begin{verbatim}
[1]   1   4   9  16  25  36  49  64  81 100
\end{verbatim}

\lstset{language=r,label= ,caption= ,captionpos=b,numbers=none}
\begin{lstlisting}
power(1:10, exp=2)
\end{lstlisting}

\begin{verbatim}
[1]   1   4   9  16  25  36  49  64  81 100
\end{verbatim}

\lstset{language=r,label= ,caption= ,captionpos=b,numbers=none}
\begin{lstlisting}
power(exp=2, x=1:10)
\end{lstlisting}

\begin{verbatim}
[1]   1   4   9  16  25  36  49  64  81 100
\end{verbatim}
\end{frame}

\begin{frame}[fragile,label={sec:orga69552a}]{Argumentos: valores por defecto}
 \begin{itemize}
\item Se puede asignar un valor por defecto a los argumentos
\end{itemize}
\lstset{language=r,label= ,caption= ,captionpos=b,numbers=none}
\begin{lstlisting}
power <- function(x, exp = 2)
{
    x ^ exp
}
\end{lstlisting}

\lstset{language=r,label= ,caption= ,captionpos=b,numbers=none}
\begin{lstlisting}
power(1:10)
\end{lstlisting}

\begin{verbatim}
[1]   1   4   9  16  25  36  49  64  81 100
\end{verbatim}

\lstset{language=r,label= ,caption= ,captionpos=b,numbers=none}
\begin{lstlisting}
power(1:10, 2)
\end{lstlisting}

\begin{verbatim}
[1]   1   4   9  16  25  36  49  64  81 100
\end{verbatim}
\end{frame}

\begin{frame}[fragile,label={sec:org48756a6}]{Funciones sin argumentos}
 \lstset{language=r,label= ,caption= ,captionpos=b,numbers=none}
\begin{lstlisting}
hello <- function()
{
    print('Hello world!')
}
\end{lstlisting}

\lstset{language=r,label= ,caption= ,captionpos=b,numbers=none}
\begin{lstlisting}
hello()
\end{lstlisting}

\begin{verbatim}
[1] "Hello world!"
\end{verbatim}
\end{frame}

\begin{frame}[fragile,label={sec:org05adb7a}]{Argumentos sin nombre: \texttt{...}}
 \lstset{language=r,label= ,caption= ,captionpos=b,numbers=none}
\begin{lstlisting}
pwrSum <- function(x, power, ...)
{
    sum(x ^ power, ...)
}
\end{lstlisting}

\lstset{language=r,label= ,caption= ,captionpos=b,numbers=none}
\begin{lstlisting}
x <- 1:10
pwrSum(x, 2)
\end{lstlisting}

\begin{verbatim}
[1] 385
\end{verbatim}

\lstset{language=r,label= ,caption= ,captionpos=b,numbers=none}
\begin{lstlisting}
x <- c(1:5, NA, 6:9, NA, 10)
pwrSum(x, 2)
\end{lstlisting}

\begin{verbatim}
[1] NA
\end{verbatim}

\lstset{language=r,label= ,caption= ,captionpos=b,numbers=none}
\begin{lstlisting}
pwrSum(x, 2, na.rm=TRUE)
\end{lstlisting}

\begin{verbatim}
[1] 385
\end{verbatim}
\end{frame}

\begin{frame}[fragile,label={sec:org3925900}]{Argumentos ausentes: \texttt{missing}}
 \lstset{language=r,label= ,caption= ,captionpos=b,numbers=none}
\begin{lstlisting}
suma10 <- function(x, y)
{
    if (missing(y)) y <- 10
    x + y
}
\end{lstlisting}

\lstset{language=r,label= ,caption= ,captionpos=b,numbers=none}
\begin{lstlisting}
suma10(1:10)
\end{lstlisting}

\begin{verbatim}
[1] 11 12 13 14 15 16 17 18 19 20
\end{verbatim}
\end{frame}

\begin{frame}[fragile,label={sec:org2d08a55}]{Control de errores: \texttt{stopifnot}}
 \lstset{language=r,label= ,caption= ,captionpos=b,numbers=none}
\begin{lstlisting}
foo <- function(x, y)
{
    stopifnot(is.numeric(x) & is.numeric(y))
    x + y
}
\end{lstlisting}

\lstset{language=r,label= ,caption= ,captionpos=b,numbers=none}
\begin{lstlisting}
foo(1:10, 21:30)
\end{lstlisting}

\begin{verbatim}
[1] 22 24 26 28 30 32 34 36 38 40
\end{verbatim}

\lstset{language=r,label= ,caption= ,captionpos=b,numbers=none}
\begin{lstlisting}
foo(1:10, 'a')
\end{lstlisting}

\begin{verbatim}
Error: is.numeric(x) & is.numeric(y) is not TRUE
\end{verbatim}
\end{frame}

\begin{frame}[fragile,label={sec:org72c445e}]{Control de errores: \texttt{stop}}
 \lstset{language=r,label= ,caption= ,captionpos=b,numbers=none}
\begin{lstlisting}
foo <- function(x, y){
    if (!(is.numeric(x) & is.numeric(y))){
        stop('arguments must be numeric.')
    } else { x + y }
} 
\end{lstlisting}

\lstset{language=r,label= ,caption= ,captionpos=b,numbers=none}
\begin{lstlisting}
foo(2, 3)
\end{lstlisting}

\begin{verbatim}
[1] 5
\end{verbatim}

\lstset{language=r,label= ,caption= ,captionpos=b,numbers=none}
\begin{lstlisting}
foo(2, 'a')
\end{lstlisting}

\begin{verbatim}
Error in foo(2, "a") : arguments must be numeric.
\end{verbatim}
\end{frame}

\begin{frame}[fragile,label={sec:orgd7190ff}]{Mensajes para el usuario}
 \texttt{stop} para la ejecución y emite un mensaje de error
\lstset{language=r,label= ,caption= ,captionpos=b,numbers=none}
\begin{lstlisting}
stop('Algo no ha ido bien.')
\end{lstlisting}

\begin{verbatim}
Error: Algo no ha ido bien.
\end{verbatim}

\texttt{warning} no interfiere en la ejecución pero añade un mensaje a la cola de advertencias
\lstset{language=r,label= ,caption= ,captionpos=b,numbers=none}
\begin{lstlisting}
warning('Quizás algo no es como debiera...')
\end{lstlisting}

\begin{verbatim}
Warning message:
Quizás algo no es como debiera...
\end{verbatim}

\texttt{message} emite un mensaje (\alert{no usar \texttt{cat} o \texttt{print}})
\lstset{language=r,label= ,caption= ,captionpos=b,numbers=none}
\begin{lstlisting}
message('Todo en orden por estos lares.')
\end{lstlisting}

\begin{verbatim}
Todo en orden por estos lares.
\end{verbatim}
\end{frame}

\section{Lexical scope}
\label{sec:orgff27276}

\begin{frame}[fragile,label={sec:org54c26da}]{Clases de variables}
 Las variables que se emplean en el cuerpo de una función pueden
dividirse en:
\begin{itemize}
\item Parámetros formales (argumentos): \texttt{x}, \texttt{y}
\item Variables locales (definiciones internas): \texttt{z}, \texttt{w}, \texttt{m}
\item Variables libres: \texttt{a}, \texttt{b}
\end{itemize}
\lstset{language=r,label= ,caption= ,captionpos=b,numbers=none}
\begin{lstlisting}
myFun <- function(x, y){
    z <- x^2
    w <- y^3
    m <- a*z + b*w
    m
}
\end{lstlisting}

\lstset{language=r,label= ,caption= ,captionpos=b,numbers=none}
\begin{lstlisting}
a <- 10
b <- 20
myFun(2, 3)
\end{lstlisting}

\begin{verbatim}
[1] 580
\end{verbatim}
\end{frame}

\begin{frame}[fragile,label={sec:org9e8cac1}]{Lexical scope}
 \begin{itemize}
\item Las variables libres deben estar disponibles en el entorno
(\texttt{environment}) en el que la función ha sido creada.
\end{itemize}
\lstset{language=r,label= ,caption= ,captionpos=b,numbers=none}
\begin{lstlisting}
environment(myFun) 
\end{lstlisting}

\begin{verbatim}
<environment: R_GlobalEnv>
\end{verbatim}

\lstset{language=r,label= ,caption= ,captionpos=b,numbers=none}
\begin{lstlisting}
ls()
\end{lstlisting}

\begin{verbatim}
[1] "a"      "b"      "foo"    "hello"  "myFun"  "power"  "pwrSum" "suma10"
[9] "x"
\end{verbatim}
\end{frame}

\begin{frame}[fragile,label={sec:org893a61b}]{Lexical scope: funciones anidadas}
 \lstset{language=r,label= ,caption= ,captionpos=b,numbers=none}
\begin{lstlisting}
anidada <- function(x, y){
    xn <- 2
    yn <- 3
    interna <- function(x, y)
    {
        sum(x^xn, y^yn)
    }
    print(environment(interna))
    interna(x, y)
}
\end{lstlisting}

\lstset{language=r,label= ,caption= ,captionpos=b,numbers=none}
\begin{lstlisting}
anidada(1:3, 2:4)
\end{lstlisting}

\begin{verbatim}
<environment: 0x563619cd90d8>
[1] 113
\end{verbatim}

\lstset{language=r,label= ,caption= ,captionpos=b,numbers=none}
\begin{lstlisting}
sum((1:3)^2, (2:4)^3)
\end{lstlisting}

\begin{verbatim}
[1] 113
\end{verbatim}
\end{frame}

\begin{frame}[fragile,label={sec:orgb740c0c}]{Lexical scope: funciones anidadas}
 \lstset{language=r,label= ,caption= ,captionpos=b,numbers=none}
\begin{lstlisting}
xn
\end{lstlisting}

\begin{verbatim}
Error: objeto 'xn' no encontrado
\end{verbatim}

\lstset{language=r,label= ,caption= ,captionpos=b,numbers=none}
\begin{lstlisting}
yn
\end{lstlisting}

\begin{verbatim}
Error: objeto 'yn' no encontrado
\end{verbatim}

\lstset{language=r,label= ,caption= ,captionpos=b,numbers=none}
\begin{lstlisting}
interna
\end{lstlisting}

\begin{verbatim}
Error: objeto 'interna' no encontrado
\end{verbatim}
\end{frame}

\begin{frame}[fragile,label={sec:orgb0922e3}]{Funciones que devuelven funciones}
 \lstset{language=r,label= ,caption= ,captionpos=b,numbers=none}
\begin{lstlisting}
constructor <- function(m, n){
    function(x)
    {
        m*x + n
    }
}
\end{lstlisting}

\lstset{language=r,label= ,caption= ,captionpos=b,numbers=none}
\begin{lstlisting}
myFoo <- constructor(10, 3)
myFoo
\end{lstlisting}

\begin{verbatim}
function(x)
    {
        m*x + n
    }
<environment: 0x563619cde4e8>
\end{verbatim}

\lstset{language=r,label= ,caption= ,captionpos=b,numbers=none}
\begin{lstlisting}
## 10*5 + 3
myFoo(5)
\end{lstlisting}

\begin{verbatim}
[1] 53
\end{verbatim}
\end{frame}

\begin{frame}[fragile,label={sec:orgc6a2b3b}]{Funciones que devuelven funciones}
 \lstset{language=r,label= ,caption= ,captionpos=b,numbers=none}
\begin{lstlisting}
class(myFoo)
\end{lstlisting}

\begin{verbatim}
[1] "function"
\end{verbatim}

\lstset{language=r,label= ,caption= ,captionpos=b,numbers=none}
\begin{lstlisting}
environment(myFoo)
\end{lstlisting}

\begin{verbatim}
<environment: 0x563619cde4e8>
\end{verbatim}

\lstset{language=r,label= ,caption= ,captionpos=b,numbers=none}
\begin{lstlisting}
ls()
\end{lstlisting}

\begin{verbatim}
 [1] "a"           "anidada"     "b"           "constructor" "foo"        
 [6] "hello"       "myFoo"       "myFun"       "power"       "pwrSum"     
[11] "suma10"      "x"
\end{verbatim}

\lstset{language=r,label= ,caption= ,captionpos=b,numbers=none}
\begin{lstlisting}
ls(env = environment(myFoo))
\end{lstlisting}

\begin{verbatim}
[1] "m" "n"
\end{verbatim}

\lstset{language=r,label= ,caption= ,captionpos=b,numbers=none}
\begin{lstlisting}
get('m', env = environment(myFoo))
\end{lstlisting}

\begin{verbatim}
[1] 10
\end{verbatim}

\lstset{language=r,label= ,caption= ,captionpos=b,numbers=none}
\begin{lstlisting}
get('n', env = environment(myFoo))
\end{lstlisting}

\begin{verbatim}
[1] 3
\end{verbatim}
\end{frame}

\section{Debug}
\label{sec:orge480d2c}

\begin{frame}[fragile,label={sec:org37cc471}]{Post-mortem: \texttt{traceback}}
 \lstset{language=r,label= ,caption= ,captionpos=b,numbers=none}
\begin{lstlisting}
sumSq <- function(x, ...){
    sum(x ^ 2, ...)
}

sumProd <- function(x, y, ...){
    xs <- sumSq(x, ...)
    ys <- sumSq(y, ...)
    xs * ys
}
\end{lstlisting}

\lstset{language=r,label= ,caption= ,captionpos=b,numbers=none}
\begin{lstlisting}
sumProd(rnorm(10), runif(10))
\end{lstlisting}

\begin{verbatim}
[1] 22.99016
\end{verbatim}

\lstset{language=r,label= ,caption= ,captionpos=b,numbers=none}
\begin{lstlisting}
sumProd(rnorm(10), letters[1:10])
\end{lstlisting}

\begin{verbatim}
Error in x^2 : argumento no-numérico para operador binario
\end{verbatim}

\lstset{language=r,label= ,caption= ,captionpos=b,numbers=none}
\begin{lstlisting}
traceback()
\end{lstlisting}

\begin{verbatim}
2: sumSq(y, ...) at #3
1: sumProd(rnorm(10), letters[1:10])
\end{verbatim}
\end{frame}

\begin{frame}[fragile,label={sec:orge3824e9}]{Analizar antes de que ocurra: \texttt{debug}}
 \begin{itemize}
\item Activa la ejecución paso a paso de una función
\end{itemize}
\lstset{language=r,label= ,caption= ,captionpos=b,numbers=none}
\begin{lstlisting}
debug(sumProd)
\end{lstlisting}

\begin{itemize}
\item Cada vez que se llame a la función, su cuerpo se ejecuta línea a línea y los resultados de cada paso pueden ser inspeccionados.
\item Los comandos disponibles son:
\begin{itemize}
\item \texttt{n} o intro: avanzar un paso.
\item \texttt{c}: continua hasta el final del contexto actual (por ejemplo,
terminar un bucle).
\item \texttt{where}: entrega la lista de todas las llamadas activas.
\item \texttt{Q}: termina la inspección y vuelve al nivel superior.
\end{itemize}
\item Para desactivar el análisis:
\end{itemize}
\lstset{language=r,label= ,caption= ,captionpos=b,numbers=none}
\begin{lstlisting}
undebug(sumProd)
\end{lstlisting}
\end{frame}

\begin{frame}[fragile,label={sec:org2973146}]{Analizar antes de que ocurra: \texttt{trace}}
 \begin{itemize}
\item \texttt{trace} permite mayor control que \texttt{debug}
\end{itemize}
\lstset{language=r,label= ,caption= ,captionpos=b,numbers=none}
\begin{lstlisting}
trace(sumProd, tracer=browser, exit=browser)
\end{lstlisting}

\begin{verbatim}
[1] "sumProd"
\end{verbatim}

\begin{itemize}
\item La función queda modificada
\end{itemize}
\lstset{language=r,label= ,caption= ,captionpos=b,numbers=none}
\begin{lstlisting}
sumProd
\end{lstlisting}

\begin{verbatim}
Object with tracing code, class "functionWithTrace"
Original definition: 
function(x, y, ...){
    xs <- sumSq(x, ...)
    ys <- sumSq(y, ...)
    xs * ys
}
<bytecode: 0x563619ee0910>

## (to see the tracing code, look at body(object))
\end{verbatim}

\lstset{language=r,label= ,caption= ,captionpos=b,numbers=none}
\begin{lstlisting}
body(sumProd)
\end{lstlisting}

\begin{verbatim}
{
    on.exit(.doTrace(browser(), "on exit"))
    {
        .doTrace(browser(), "on entry")
        {
            xs <- sumSq(x, ...)
            ys <- sumSq(y, ...)
            xs * ys
        }
    }
}
\end{verbatim}
\end{frame}

\begin{frame}[fragile,label={sec:orgfff0947}]{Analizar antes de que ocurra: \texttt{trace}}
 \begin{itemize}
\item Los comandos \texttt{n} y \texttt{c} cambian respecto a \texttt{debug}:
\begin{itemize}
\item \texttt{c} o intro: avanzar un paso.
\item \texttt{n}: continua hasta el final del contexto actual (por ejemplo,
terminar un bucle).
\end{itemize}
\item Para desactivar
\end{itemize}
\lstset{language=r,label= ,caption= ,captionpos=b,numbers=none}
\begin{lstlisting}
untrace(sumProd)
\end{lstlisting}
\end{frame}

\begin{frame}[label={sec:org8009fe2}]{Más recursos}
\begin{itemize}
\item Debugging en RStudio
\begin{itemize}
\item \href{https://support.rstudio.com/hc/en-us/articles/205612627-Debugging-with-RStudio}{Artículo}
\item \href{https://vimeo.com/97831988}{Vídeo}
\end{itemize}
\item \href{http://adv-r.had.co.nz/Exceptions-Debugging.html}{\emph{Debugging} explicado por H. Wickham}
\end{itemize}
\end{frame}

\section{Profiling}
\label{sec:org0b2d786}
\begin{frame}[fragile,label={sec:orgb306455}]{¿Cuánto tarda mi función? \texttt{system.time}}
 \lstset{language=r,label= ,caption= ,captionpos=b,numbers=none}
\begin{lstlisting}
noise <- function(sd)rnorm(1000, mean=0, sd=sd)
\end{lstlisting}

\lstset{language=r,label= ,caption= ,captionpos=b,numbers=none}
\begin{lstlisting}
sumNoise <- function(nComponents){
    vals <- sapply(seq_len(nComponents), noise)
    rowSums(vals)
}
\end{lstlisting}

\lstset{language=r,label= ,caption= ,captionpos=b,numbers=none}
\begin{lstlisting}
system.time(sumNoise(1000))
\end{lstlisting}

\begin{verbatim}
 user  system elapsed 
0.125   0.003   0.128
\end{verbatim}
\end{frame}

\begin{frame}[fragile,label={sec:org4213693}]{¿Cuánto tarda cada parte de mi función?: \texttt{Rprof}}
 \begin{itemize}
\item Usaremos un fichero temporal
\end{itemize}
\lstset{language=r,label= ,caption= ,captionpos=b,numbers=none}
\begin{lstlisting}
tmp <- tempfile()
\end{lstlisting}

\begin{itemize}
\item Activamos la toma de información
\end{itemize}
\lstset{language=r,label= ,caption= ,captionpos=b,numbers=none}
\begin{lstlisting}
Rprof(tmp)
\end{lstlisting}

\begin{itemize}
\item Ejecutamos el código a analizar
\end{itemize}
\lstset{language=r,label= ,caption= ,captionpos=b,numbers=none}
\begin{lstlisting}
zz <- sumNoise(1000)
\end{lstlisting}
\end{frame}

\begin{frame}[fragile,label={sec:orgf1c99cd}]{¿Cuánto tarda cada parte de mi función?: \texttt{Rprof}}
 \begin{itemize}
\item Paramos el análisis
\end{itemize}
\lstset{language=r,label= ,caption= ,captionpos=b,numbers=none}
\begin{lstlisting}
Rprof()
\end{lstlisting}

\begin{itemize}
\item Extraemos el resumen
\end{itemize}
\lstset{language=r,label= ,caption= ,captionpos=b,numbers=none}
\begin{lstlisting}
summaryRprof(tmp)
\end{lstlisting}

\begin{verbatim}
$by.self
         self.time self.pct total.time total.pct
"rnorm"       0.06       75       0.06        75
"unlist"      0.02       25       0.02        25

$by.total
                 total.time total.pct self.time self.pct
"sapply"               0.08       100      0.00        0
"sumNoise"             0.08       100      0.00        0
"rnorm"                0.06        75      0.06       75
"FUN"                  0.06        75      0.00        0
"lapply"               0.06        75      0.00        0
"unlist"               0.02        25      0.02       25
"simplify2array"       0.02        25      0.00        0

$sample.interval
[1] 0.02

$sampling.time
[1] 0.08
\end{verbatim}
\end{frame}


\section{Miscelánea}
\label{sec:orgde3aa07}
\begin{frame}[fragile,label={sec:org423f4a9}]{\texttt{do.call}}
 \begin{itemize}
\item Ejemplo: sumar los componentes de una lista
\end{itemize}
\lstset{language=r,label= ,caption= ,captionpos=b,numbers=none}
\begin{lstlisting}
lista <- list(a = rnorm(100),
              b = runif(100),
              c = rexp(100))
with(lista, sum(a + b + c))
\end{lstlisting}

\begin{verbatim}
[1] 143.0742
\end{verbatim}

\begin{itemize}
\item En lugar de nombrar los componentes, creamos una llamada a una
función con \texttt{do.call}
\end{itemize}
\lstset{language=r,label= ,caption= ,captionpos=b,numbers=none}
\begin{lstlisting}
do.call(sum, lista)
\end{lstlisting}

\begin{verbatim}
[1] 143.0742
\end{verbatim}
\end{frame}

\begin{frame}[fragile,label={sec:org1fdb16a}]{\texttt{do.call}}
 \begin{itemize}
\item Se emplea frecuentemente con el resultado de \texttt{lapply}
\end{itemize}
\lstset{language=r,label= ,caption= ,captionpos=b,numbers=none}
\begin{lstlisting}
  x <- rnorm(5)
  ll <- lapply(1:5, function(i)x^i)
  do.call(rbind, ll)
\end{lstlisting}

\begin{verbatim}
          [,1]        [,2]       [,3]        [,4]      [,5]
[1,] 0.6353627 -0.49157083  -1.598539 -0.59264536 -1.566159
[2,] 0.4036858  0.24164188   2.555326  0.35122853  2.452854
[3,] 0.2564869 -0.11878410  -4.084788 -0.20815396 -3.841559
[4,] 0.1629622  0.05839080   6.529692  0.12336148  6.016492
[5,] 0.1035401 -0.02870321 -10.437967 -0.07310961 -9.422782
\end{verbatim}

\begin{itemize}
\item Este mismo ejemplo puede resolverse con \texttt{sapply}
\end{itemize}
\lstset{language=r,label= ,caption= ,captionpos=b,numbers=none}
\begin{lstlisting}
  sapply(1:5, function(i)x^i)
\end{lstlisting}

\begin{verbatim}
           [,1]      [,2]       [,3]      [,4]         [,5]
[1,]  0.6353627 0.4036858  0.2564869 0.1629622   0.10354013
[2,] -0.4915708 0.2416419 -0.1187841 0.0583908  -0.02870321
[3,] -1.5985388 2.5553263 -4.0847882 6.5296925 -10.43796674
[4,] -0.5926454 0.3512285 -0.2081540 0.1233615  -0.07310961
[5,] -1.5661589 2.4528538 -3.8415588 6.0164917  -9.42278210
\end{verbatim}
\end{frame}

\begin{frame}[fragile,label={sec:orgc0cba00}]{\texttt{Reduce}}
 \begin{itemize}
\item Combina sucesivamente los elementos de un objeto aplicando una
función binaria
\end{itemize}
\lstset{language=r,label= ,caption= ,captionpos=b,numbers=none}
\begin{lstlisting}
## (((1+2)+3)+4)+5
Reduce('+', 1:5)
\end{lstlisting}

\begin{verbatim}
[1] 15
\end{verbatim}
\end{frame}

\begin{frame}[fragile,label={sec:org56fd2bd}]{\texttt{Reduce}}
 \lstset{language=r,label= ,caption= ,captionpos=b,numbers=none}
\begin{lstlisting}
## (((1/2)/3)/4)/5
Reduce('/', 1:5)
\end{lstlisting}

\begin{verbatim}
[1] 0.008333333
\end{verbatim}

\lstset{language=r,label= ,caption= ,captionpos=b,numbers=none}
\begin{lstlisting}
foo <- function(u, v)u + 1 /v
Reduce(foo, c(3, 7, 15, 1, 292))
## equivalente a
## foo(foo(foo(foo(3, 7), 15), 1), 292)
\end{lstlisting}

\begin{verbatim}
[1] 4.212948
\end{verbatim}

\lstset{language=r,label= ,caption= ,captionpos=b,numbers=none}
\begin{lstlisting}
Reduce(foo, c(3, 7, 15, 1, 292), right=TRUE)
## equivalente a
## foo(3, foo(7, foo(15, foo(1, 292))))
\end{lstlisting}

\begin{verbatim}
[1] 3.141593
\end{verbatim}
\end{frame}

\begin{frame}[fragile,label={sec:org3cf7c05}]{Funciones recursivas}
 \begin{itemize}
\item \href{http://en.wikibooks.org/wiki/R\_Programming/Working\_with\_functions\#Functions\_as\_Objects}{Serie de Fibonnaci}
\end{itemize}
\lstset{language=r,label= ,caption= ,captionpos=b,numbers=none}
\begin{lstlisting}
fib <- function(n){
    if (n>2) {
        c(fib(n-1),
          sum(tail(fib(n-1),2)))
    } else if (n>=0) rep(1,n)
}
\end{lstlisting}

\lstset{language=r,label= ,caption= ,captionpos=b,numbers=none}
\begin{lstlisting}
fib(10)
\end{lstlisting}

\begin{verbatim}
[1]  1  1  2  3  5  8 13 21 34 55
\end{verbatim}
\end{frame}
\end{document}