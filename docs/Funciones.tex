% Created 2020-02-17 lun 23:23
% Intended LaTeX compiler: pdflatex
\documentclass[xcolor={usenames,svgnames,dvipsnames}]{beamer}
\usepackage[utf8]{inputenc}
\usepackage[T1]{fontenc}
\usepackage{graphicx}
\usepackage{grffile}
\usepackage{longtable}
\usepackage{wrapfig}
\usepackage{rotating}
\usepackage[normalem]{ulem}
\usepackage{amsmath}
\usepackage{textcomp}
\usepackage{amssymb}
\usepackage{capt-of}
\usepackage{hyperref}
\usepackage{color}
\usepackage{listings}
\usepackage[spanish]{babel}
\setbeamercolor{alerted text}{fg=Blue}
\setbeamerfont{alerted text}{series=\bfseries}
\setbeamercolor{block title}{bg=structure.fg!20!bg!50!bg}
\setbeamercolor{block body}{use=block title,bg=block title.bg}
\AtBeginSubsection[]{\begin{frame}[plain]\tableofcontents[currentsubsection,sectionstyle=show/shaded,subsectionstyle=show/shaded/hide]\end{frame}}
\AtBeginSection[]{\begin{frame}[plain]\tableofcontents[currentsection,hideallsubsections]\end{frame}}
\lstset{keywordstyle=\color{blue}, commentstyle=\color{gray!90}, basicstyle=\ttfamily\small, columns=fullflexible, breaklines=true,linewidth=\textwidth, backgroundcolor=\color{gray!23}, basewidth={0.5em,0.4em}, literate={á}{{\'a}}1 {ñ}{{\~n}}1 {é}{{\'e}}1 {ó}{{\'o}}1 {º}{{\textordmasculine}}1, showstringspaces=false}
\usepackage{mathpazo}
\hypersetup{colorlinks=true, linkcolor=Blue, urlcolor=Blue}
\usepackage{fancyvrb}
\DefineVerbatimEnvironment{verbatim}{Verbatim}{fontsize=\tiny, formatcom = {\color{black!70}}}
\beamertemplatenavigationsymbolsempty
\setbeamertemplate{footline}[frame number]
\AtBeginSection[]{\begin{frame}[plain]\tableofcontents[currentsection,sectionstyle=show/shaded]\end{frame}}
\usetheme{Boadilla}
\usecolortheme{rose}
\usefonttheme{serif}
\author{\href{https://oscarperpinan.github.io}{Oscar Perpiñán Lamigueiro}}
\date{}
\title{Funciones}
\institute[UPM]{Universidad Politécnica de Madrid}
\hypersetup{
 pdfauthor={\href{https://oscarperpinan.github.io}{Oscar Perpiñán Lamigueiro}},
 pdftitle={Funciones},
 pdfkeywords={},
 pdfsubject={},
 pdfcreator={Emacs 26.1 (Org mode 9.3.4)}, 
 pdflang={Spanish}}
\begin{document}

\maketitle


\section{Conceptos Básicos}
\label{sec:org807d780}

\begin{frame}[label={sec:org7b974ac}]{Fuentes de información}
\begin{itemize}
\item \href{http://cran.r-project.org/doc/manuals/R-intro.html}{R introduction}
\item \href{http://cran.r-project.org/doc/manuals/R-lang.html}{R Language Definition}
\item \href{http://www.springer.com/gb/book/9780387759357}{Software for Data Analysis}
\end{itemize}
\end{frame}
\begin{frame}[label={sec:orgd31c708},fragile]{Componentes de una función}
 \begin{itemize}
\item Una función se define con \texttt{function}
\end{itemize}
\begin{center}
\texttt{name <- function(arg\_1, arg\_2, ...) expression}
\end{center}
\begin{itemize}
\item Está compuesta por:
\begin{itemize}
\item Nombre de la función (\texttt{name})
\item Argumentos (\texttt{arg\_1}, \texttt{arg\_2}, \texttt{...})
\item Cuerpo (\texttt{expression}): emplea los argumentos para generar un resultado
\end{itemize}
\end{itemize}
\end{frame}
\begin{frame}[label={sec:orgfa16a57},fragile]{Mi primera función}
 \begin{itemize}
\item Definición
\end{itemize}
\lstset{language=r,label= ,caption= ,captionpos=b,numbers=none}
\begin{lstlisting}
myFun <- function(x, y)
{
    x + y
}
\end{lstlisting}

\begin{itemize}
\item Argumentos
\end{itemize}
\lstset{language=r,label= ,caption= ,captionpos=b,numbers=none}
\begin{lstlisting}
formals(myFun)
\end{lstlisting}

\begin{verbatim}
$x


$y
\end{verbatim}


\begin{itemize}
\item Cuerpo
\end{itemize}
\lstset{language=r,label= ,caption= ,captionpos=b,numbers=none}
\begin{lstlisting}
body(myFun)
\end{lstlisting}

\begin{verbatim}
{
    x
y
}
\end{verbatim}
\end{frame}

\begin{frame}[label={sec:orgd84b1d2},fragile]{Mi primera función}
 \lstset{language=r,label= ,caption= ,captionpos=b,numbers=none}
\begin{lstlisting}
myFun(1, 2)
\end{lstlisting}

\begin{verbatim}
[1] 3
\end{verbatim}


\lstset{language=r,label= ,caption= ,captionpos=b,numbers=none}
\begin{lstlisting}
myFun(1:10, 21:30)
\end{lstlisting}

\begin{verbatim}
[1] 22 24 26 28 30 32 34 36 38 40
\end{verbatim}


\lstset{language=r,label= ,caption= ,captionpos=b,numbers=none}
\begin{lstlisting}
myFun(1:10, 3)
\end{lstlisting}

\begin{verbatim}
[1]  4  5  6  7  8  9 10 11 12 13
\end{verbatim}
\end{frame}

\begin{frame}[label={sec:orgbe93e06},fragile]{Argumentos: nombre y orden}
 Una función identifica sus argumentos por su nombre y por su orden (sin nombre)

\lstset{language=r,label= ,caption= ,captionpos=b,numbers=none}
\begin{lstlisting}
power <- function(x, exp)
{
    x^exp
}
\end{lstlisting}

\lstset{language=r,label= ,caption= ,captionpos=b,numbers=none}
\begin{lstlisting}
power(x=1:10, exp=2)
\end{lstlisting}

\begin{verbatim}
[1]   1   4   9  16  25  36  49  64  81 100
\end{verbatim}


\lstset{language=r,label= ,caption= ,captionpos=b,numbers=none}
\begin{lstlisting}
power(1:10, exp=2)
\end{lstlisting}

\begin{verbatim}
[1]   1   4   9  16  25  36  49  64  81 100
\end{verbatim}


\lstset{language=r,label= ,caption= ,captionpos=b,numbers=none}
\begin{lstlisting}
power(exp=2, x=1:10)
\end{lstlisting}

\begin{verbatim}
[1]   1   4   9  16  25  36  49  64  81 100
\end{verbatim}
\end{frame}

\begin{frame}[label={sec:org6c422f9},fragile]{Argumentos: valores por defecto}
 \begin{itemize}
\item Se puede asignar un valor por defecto a los argumentos
\end{itemize}
\lstset{language=r,label= ,caption= ,captionpos=b,numbers=none}
\begin{lstlisting}
power <- function(x, exp = 2)
{
    x ^ exp
}
\end{lstlisting}

\lstset{language=r,label= ,caption= ,captionpos=b,numbers=none}
\begin{lstlisting}
power(1:10)
\end{lstlisting}

\begin{verbatim}
[1]   1   4   9  16  25  36  49  64  81 100
\end{verbatim}


\lstset{language=r,label= ,caption= ,captionpos=b,numbers=none}
\begin{lstlisting}
power(1:10, 2)
\end{lstlisting}

\begin{verbatim}
[1]   1   4   9  16  25  36  49  64  81 100
\end{verbatim}
\end{frame}

\begin{frame}[label={sec:org0458aec},fragile]{Funciones sin argumentos}
 \lstset{language=r,label= ,caption= ,captionpos=b,numbers=none}
\begin{lstlisting}
hello <- function()
{
    print('Hello world!')
}
\end{lstlisting}

\lstset{language=r,label= ,caption= ,captionpos=b,numbers=none}
\begin{lstlisting}
hello()
\end{lstlisting}

\begin{verbatim}
[1] "Hello world!"
\end{verbatim}
\end{frame}

\begin{frame}[label={sec:orgf53235a},fragile]{Argumentos sin nombre: \texttt{...}}
 \lstset{language=r,label= ,caption= ,captionpos=b,numbers=none}
\begin{lstlisting}
pwrSum <- function(x, power, ...)
{
    sum(x ^ power, ...)
}
\end{lstlisting}

\lstset{language=r,label= ,caption= ,captionpos=b,numbers=none}
\begin{lstlisting}
x <- 1:10
pwrSum(x, 2)
\end{lstlisting}

\begin{verbatim}

[1] 385
\end{verbatim}


\lstset{language=r,label= ,caption= ,captionpos=b,numbers=none}
\begin{lstlisting}
x <- c(1:5, NA, 6:9, NA, 10)
pwrSum(x, 2)
\end{lstlisting}

\begin{verbatim}

[1] NA
\end{verbatim}


\lstset{language=r,label= ,caption= ,captionpos=b,numbers=none}
\begin{lstlisting}
pwrSum(x, 2, na.rm=TRUE)
\end{lstlisting}

\begin{verbatim}
[1] 385
\end{verbatim}
\end{frame}

\begin{frame}[label={sec:orgca020c7},fragile]{Argumentos ausentes: \texttt{missing}}
 \lstset{language=r,label= ,caption= ,captionpos=b,numbers=none}
\begin{lstlisting}
suma10 <- function(x, y)
{
    if (missing(y)) y <- 10
    x + y
}
\end{lstlisting}

\lstset{language=r,label= ,caption= ,captionpos=b,numbers=none}
\begin{lstlisting}
suma10(1:10)
\end{lstlisting}

\begin{verbatim}
[1] 11 12 13 14 15 16 17 18 19 20
\end{verbatim}
\end{frame}

\begin{frame}[label={sec:orgf5e922a},fragile]{Control de errores: \texttt{stopifnot}}
 \lstset{language=r,label= ,caption= ,captionpos=b,numbers=none}
\begin{lstlisting}
foo <- function(x, y)
{
    stopifnot(is.numeric(x) & is.numeric(y))
    x + y
}
\end{lstlisting}

\lstset{language=r,label= ,caption= ,captionpos=b,numbers=none}
\begin{lstlisting}
foo(1:10, 21:30)
\end{lstlisting}

\begin{verbatim}
[1] 22 24 26 28 30 32 34 36 38 40
\end{verbatim}


\lstset{language=r,label= ,caption= ,captionpos=b,numbers=none}
\begin{lstlisting}
foo(1:10, 'a')
\end{lstlisting}

\begin{verbatim}
Error in foo(1:10, "a") : is.numeric(x) & is.numeric(y) is not TRUE
\end{verbatim}
\end{frame}

\begin{frame}[label={sec:orge737c5c},fragile]{Control de errores: \texttt{stop}}
 \lstset{language=r,label= ,caption= ,captionpos=b,numbers=none}
\begin{lstlisting}
foo <- function(x, y){
    if (!(is.numeric(x) & is.numeric(y))){
        stop('arguments must be numeric.')
    } else { x + y }
} 
\end{lstlisting}

\lstset{language=r,label= ,caption= ,captionpos=b,numbers=none}
\begin{lstlisting}
foo(2, 3)
\end{lstlisting}

\begin{verbatim}
[1] 5
\end{verbatim}


\lstset{language=r,label= ,caption= ,captionpos=b,numbers=none}
\begin{lstlisting}
foo(2, 'a')
\end{lstlisting}

\begin{verbatim}
Error in foo(2, "a") : arguments must be numeric.
\end{verbatim}
\end{frame}

\begin{frame}[label={sec:org0f4b548},fragile]{Mensajes para el usuario}
 \texttt{stop} para la ejecución y emite un mensaje de error
\lstset{language=r,label= ,caption= ,captionpos=b,numbers=none}
\begin{lstlisting}
stop('Algo no ha ido bien.')
\end{lstlisting}

\begin{verbatim}
Error: Algo no ha ido bien.
\end{verbatim}


\texttt{warning} no interfiere en la ejecución pero añade un mensaje a la cola de advertencias
\lstset{language=r,label= ,caption= ,captionpos=b,numbers=none}
\begin{lstlisting}
warning('Quizás algo no es como debiera...')
\end{lstlisting}

\begin{verbatim}
Warning message:
Quizás algo no es como debiera...
\end{verbatim}


\texttt{message} emite un mensaje (\alert{no usar \texttt{cat} o \texttt{print}})
\lstset{language=r,label= ,caption= ,captionpos=b,numbers=none}
\begin{lstlisting}
message('Todo en orden por estos lares.')
\end{lstlisting}

\begin{verbatim}
Todo en orden por estos lares.
\end{verbatim}
\end{frame}


\section{Lexical scope}
\label{sec:orgb2e409a}

\begin{frame}[label={sec:orga32f7a1},fragile]{Clases de variables}
 Las variables que se emplean en el cuerpo de una función pueden
dividirse en:
\begin{itemize}
\item Parámetros formales (argumentos): \texttt{x}, \texttt{y}
\item Variables locales (definiciones internas): \texttt{z}, \texttt{w}, \texttt{m}
\item Variables libres: \texttt{a}, \texttt{b}
\end{itemize}
\lstset{language=r,label= ,caption= ,captionpos=b,numbers=none}
\begin{lstlisting}
myFun <- function(x, y){
    z <- x^2
    w <- y^3
    m <- a*z + b*w
    m
}
\end{lstlisting}

\lstset{language=r,label= ,caption= ,captionpos=b,numbers=none}
\begin{lstlisting}
a <- 10
b <- 20
myFun(2, 3)
\end{lstlisting}

\begin{verbatim}

[1] 580
\end{verbatim}
\end{frame}

\begin{frame}[label={sec:org1728385},fragile]{Lexical scope}
 \begin{itemize}
\item Las variables libres deben estar disponibles en el entorno
(\texttt{environment}) en el que la función ha sido creada.
\end{itemize}
\lstset{language=r,label= ,caption= ,captionpos=b,numbers=none}
\begin{lstlisting}
environment(myFun) 
\end{lstlisting}

\begin{verbatim}
<environment: R_GlobalEnv>
\end{verbatim}


\lstset{language=r,label= ,caption= ,captionpos=b,numbers=none}
\begin{lstlisting}
ls()
\end{lstlisting}

\begin{verbatim}
 [1] "a"           "anidada"     "b"           "constructor" "fib"        
 [6] "foo"         "hello"       "i"           "lista"       "ll"         
[11] "M"           "makeNoise"   "myFoo"       "myFun"       "power"      
[16] "pwrSum"      "suma1"       "suma10"      "suma2"       "suma3"      
[21] "sumProd"     "sumSq"       "tmp"         "x"
\end{verbatim}
\end{frame}

\begin{frame}[label={sec:org826acc0},fragile]{Lexical scope: funciones anidadas}
 \lstset{language=r,label= ,caption= ,captionpos=b,numbers=none}
\begin{lstlisting}
anidada <- function(x, y){
    xn <- 2
    yn <- 3
    interna <- function(x, y)
    {
        sum(x^xn, y^yn)
    }
    print(environment(interna))
    interna(x, y)
}
\end{lstlisting}

\lstset{language=r,label= ,caption= ,captionpos=b,numbers=none}
\begin{lstlisting}
anidada(1:3, 2:4)
\end{lstlisting}

\begin{verbatim}
<environment: 0x561461a60550>
[1] 113
\end{verbatim}


\lstset{language=r,label= ,caption= ,captionpos=b,numbers=none}
\begin{lstlisting}
sum((1:3)^2, (2:4)^3)
\end{lstlisting}

\begin{verbatim}
[1] 113
\end{verbatim}
\end{frame}

\begin{frame}[label={sec:orgd230cb0},fragile]{Lexical scope: funciones anidadas}
 \lstset{language=r,label= ,caption= ,captionpos=b,numbers=none}
\begin{lstlisting}
xn
\end{lstlisting}

\begin{verbatim}
Error: objeto 'xn' no encontrado
\end{verbatim}


\lstset{language=r,label= ,caption= ,captionpos=b,numbers=none}
\begin{lstlisting}
yn
\end{lstlisting}

\begin{verbatim}
Error: objeto 'yn' no encontrado
\end{verbatim}


\lstset{language=r,label= ,caption= ,captionpos=b,numbers=none}
\begin{lstlisting}
interna
\end{lstlisting}

\begin{verbatim}
Error: objeto 'interna' no encontrado
\end{verbatim}
\end{frame}

\begin{frame}[label={sec:org013a655},fragile]{Funciones que devuelven funciones}
 \lstset{language=r,label= ,caption= ,captionpos=b,numbers=none}
\begin{lstlisting}
constructor <- function(m, n){
    function(x)
    {
        m*x + n
    }
}
\end{lstlisting}

\lstset{language=r,label= ,caption= ,captionpos=b,numbers=none}
\begin{lstlisting}
myFoo <- constructor(10, 3)
myFoo
\end{lstlisting}

\begin{verbatim}

function(x)
    {
        m*x
n
    }
<environment: 0x561469bf1458>
\end{verbatim}


\lstset{language=r,label= ,caption= ,captionpos=b,numbers=none}
\begin{lstlisting}
## 10*5 + 3
myFoo(5)
\end{lstlisting}

\begin{verbatim}

[1] 53
\end{verbatim}
\end{frame}

\begin{frame}[label={sec:orgaa6227e},fragile]{Funciones que devuelven funciones}
 \lstset{language=r,label= ,caption= ,captionpos=b,numbers=none}
\begin{lstlisting}
class(myFoo)
\end{lstlisting}

\begin{verbatim}
[1] "function"
\end{verbatim}


\lstset{language=r,label= ,caption= ,captionpos=b,numbers=none}
\begin{lstlisting}
environment(myFoo)
\end{lstlisting}

\begin{verbatim}
<environment: 0x561469bf1458>
\end{verbatim}


\lstset{language=r,label= ,caption= ,captionpos=b,numbers=none}
\begin{lstlisting}
ls()
\end{lstlisting}

\begin{verbatim}
 [1] "a"           "anidada"     "b"           "constructor" "fib"        
 [6] "foo"         "hello"       "i"           "lista"       "ll"         
[11] "M"           "makeNoise"   "myFoo"       "myFun"       "power"      
[16] "pwrSum"      "suma1"       "suma10"      "suma2"       "suma3"      
[21] "sumProd"     "sumSq"       "tmp"         "x"
\end{verbatim}


\lstset{language=r,label= ,caption= ,captionpos=b,numbers=none}
\begin{lstlisting}
ls(env = environment(myFoo))
\end{lstlisting}

\begin{verbatim}
[1] "m" "n"
\end{verbatim}


\lstset{language=r,label= ,caption= ,captionpos=b,numbers=none}
\begin{lstlisting}
get('m', env = environment(myFoo))
\end{lstlisting}

\begin{verbatim}
[1] 10
\end{verbatim}


\lstset{language=r,label= ,caption= ,captionpos=b,numbers=none}
\begin{lstlisting}
get('n', env = environment(myFoo))
\end{lstlisting}

\begin{verbatim}
[1] 3
\end{verbatim}
\end{frame}

\section{Funciones para ejecutar funciones}
\label{sec:org51c9b1e}
\begin{frame}[label={sec:orga96fb82},fragile]{\texttt{lapply}}
 Supongamos que tenemos una lista de objetos, y queremos aplicar a cada elemento la misma función:
\lstset{language=r,label= ,caption= ,captionpos=b,numbers=none}
\begin{lstlisting}
lista <- list(a = rnorm(100),
              b = runif(100),
              c = rexp(100))
\end{lstlisting}

Podemos resolverlo de forma repetitiva\ldots{}
\lstset{language=r,label= ,caption= ,captionpos=b,numbers=none}
\begin{lstlisting}
sum(lista$a)

sum(lista$b)

sum(lista$c)
\end{lstlisting}

\begin{verbatim}
[1] -3.295036

[1] 51.57184

[1] 99.86313
\end{verbatim}


O mejor con \texttt{lapply} (lista + función):
\lstset{language=r,label= ,caption= ,captionpos=b,numbers=none}
\begin{lstlisting}
lapply(lista, sum)
\end{lstlisting}

\begin{verbatim}
$a
[1] -3.295036

$b
[1] 51.57184

$c
[1] 99.86313
\end{verbatim}
\end{frame}

\begin{frame}[label={sec:orga29c1f0},fragile]{\texttt{do.call}}
 Supongamos que queremos usar los elementos de la lista como argumentos de una función.

Resolvemos de forma directa:
\lstset{language=r,label= ,caption= ,captionpos=b,numbers=none}
\begin{lstlisting}
sum(lista$a, lista$b, lista$c)
\end{lstlisting}

\begin{verbatim}
[1] 148.1399
\end{verbatim}


Mejoramos \emph{un poco} con \texttt{with}:
\lstset{language=r,label= ,caption= ,captionpos=b,numbers=none}
\begin{lstlisting}
with(lista, sum(a, b, c))
\end{lstlisting}

\begin{verbatim}
[1] 148.1399
\end{verbatim}


La forma recomendable es mediante \texttt{do.call} (función + lista)
\lstset{language=r,label= ,caption= ,captionpos=b,numbers=none}
\begin{lstlisting}
do.call(sum, lista)
\end{lstlisting}

\begin{verbatim}
[1] 148.1399
\end{verbatim}
\end{frame}

\begin{frame}[label={sec:org4001958},fragile]{\texttt{do.call}}
 Se emplea frecuentemente para adecuar el resultado de \texttt{lapply} (entrega una lista):
\lstset{language=r,label= ,caption= ,captionpos=b,numbers=none}
\begin{lstlisting}
  x <- rnorm(5)
  ll <- lapply(1:5, function(i)x^i)
  do.call(rbind, ll)
\end{lstlisting}

\begin{verbatim}

          [,1]       [,2]       [,3]      [,4]      [,5]
[1,]  1.980628 -0.8595001 0.45504847 -1.469399 0.6453231
[2,]  3.922888  0.7387405 0.20706911  2.159134 0.4164419
[3,]  7.769784 -0.6349475 0.09422648 -3.172631 0.2687396
[4,] 15.389054  0.5457375 0.04287762  4.661861 0.1734239
[5,] 30.479996 -0.4690614 0.01951139 -6.850136 0.1119144
\end{verbatim}
\end{frame}

\begin{frame}[label={sec:orgd71f79d},fragile]{\texttt{Reduce}}
 Combina sucesivamente los elementos de un objeto aplicando una función binaria
\lstset{language=r,label= ,caption= ,captionpos=b,numbers=none}
\begin{lstlisting}
## (((1+2)+3)+4)+5
Reduce('+', 1:5)
\end{lstlisting}

\begin{verbatim}

[1] 15
\end{verbatim}
\end{frame}

\begin{frame}[label={sec:orgca0e686},fragile]{\texttt{Reduce}}
 \lstset{language=r,label= ,caption= ,captionpos=b,numbers=none}
\begin{lstlisting}
## (((1/2)/3)/4)/5
Reduce('/', 1:5)
\end{lstlisting}

\begin{verbatim}

[1] 0.008333333
\end{verbatim}


\lstset{language=r,label= ,caption= ,captionpos=b,numbers=none}
\begin{lstlisting}
foo <- function(u, v)u + 1 /v
Reduce(foo, c(3, 7, 15, 1, 292))
## equivalente a
## foo(foo(foo(foo(3, 7), 15), 1), 292)
\end{lstlisting}

\begin{verbatim}

[1] 4.212948
\end{verbatim}


\lstset{language=r,label= ,caption= ,captionpos=b,numbers=none}
\begin{lstlisting}
Reduce(foo, c(3, 7, 15, 1, 292), right=TRUE)
## equivalente a
## foo(3, foo(7, foo(15, foo(1, 292))))
\end{lstlisting}

\begin{verbatim}
[1] 3.141593
\end{verbatim}
\end{frame}

\begin{frame}[label={sec:org73d81ea},fragile]{Funciones recursivas}
 Ejemplo: \href{http://en.wikibooks.org/wiki/R\_Programming/Working\_with\_functions\#Functions\_as\_Objects}{Serie de Fibonnaci}
\lstset{language=r,label= ,caption= ,captionpos=b,numbers=none}
\begin{lstlisting}
fib <- function(n){
    if (n>2) {
        c(fib(n-1),
          sum(tail(fib(n-1),2)))
    } else if (n>=0) rep(1,n)
}
\end{lstlisting}

\lstset{language=r,label= ,caption= ,captionpos=b,numbers=none}
\begin{lstlisting}
fib(10)
\end{lstlisting}

\begin{verbatim}
[1]  1  1  2  3  5  8 13 21 34 55
\end{verbatim}
\end{frame}

\section{Debug}
\label{sec:org2b6fcb0}

\begin{frame}[label={sec:orga900526},fragile]{Post-mortem: \texttt{traceback}}
 \lstset{language=r,label= ,caption= ,captionpos=b,numbers=none}
\begin{lstlisting}
sumSq <- function(x, ...)
    sum(x ^ 2, ...)

sumProd <- function(x, y, ...){
    xs <- sumSq(x, ...)
    ys <- sumSq(y, ...)
    xs * ys
}
\end{lstlisting}

\lstset{language=r,label= ,caption= ,captionpos=b,numbers=none}
\begin{lstlisting}
sumProd(rnorm(10), runif(10))
\end{lstlisting}

\begin{verbatim}
[1] 8.711019
\end{verbatim}


\lstset{language=r,label= ,caption= ,captionpos=b,numbers=none}
\begin{lstlisting}
sumProd(rnorm(10), letters[1:10])
\end{lstlisting}

\begin{verbatim}
Error in x^2 : argumento no-numérico para operador binario
\end{verbatim}


\lstset{language=r,label= ,caption= ,captionpos=b,numbers=none}
\begin{lstlisting}
traceback()
\end{lstlisting}

\begin{verbatim}
2: sumSq(y, ...) at #3
1: sumProd(rnorm(10), letters[1:10])
\end{verbatim}
\end{frame}

\begin{frame}[label={sec:orgb6c7b9a},fragile]{Analizar antes de que ocurra: \texttt{debug}}
 \texttt{debug} activa la ejecución paso a paso de una función:
\lstset{language=r,label= ,caption= ,captionpos=b,numbers=none}
\begin{lstlisting}
debug(sumProd)
\end{lstlisting}

\begin{itemize}
\item Cada vez que se llame a la función, su cuerpo se ejecuta línea a línea y los resultados de cada paso pueden ser inspeccionados.
\item Los comandos disponibles son:
\begin{itemize}
\item \texttt{n} o intro: avanzar un paso.
\item \texttt{c}: continua hasta el final del contexto actual (por ejemplo,
terminar un bucle).
\item \texttt{where}: entrega la lista de todas las llamadas activas.
\item \texttt{Q}: termina la inspección y vuelve al nivel superior.
\end{itemize}
\item Para desactivar el análisis:
\end{itemize}
\lstset{language=r,label= ,caption= ,captionpos=b,numbers=none}
\begin{lstlisting}
undebug(sumProd)
\end{lstlisting}
\end{frame}

\begin{frame}[label={sec:orgfcb0a28},fragile]{\emph{Debugging} con RStudio}
 \begin{itemize}
\item \href{https://support.rstudio.com/hc/en-us/articles/205612627-Debugging-with-RStudio}{Artículo}
\item \href{https://vimeo.com/99375765https://vimeo.com/97831988}{Vídeo}
\end{itemize}
\begin{itemize}
\item \href{http://adv-r.had.co.nz/Exceptions-Debugging.html}{\emph{Debugging} explicado por H. Wickham}

\item Ejemplo: grabar en un fichero y usar \emph{source}
\end{itemize}
\lstset{language=r,label= ,caption= ,captionpos=b,numbers=none}
\begin{lstlisting}
sumSq <- function(x, ...)
    sum(x ^ 2, ...)

sumProd <- function(x, y, ...){
    xs <- sumSq(x, ...)
    ys <- sumSq(y, ...)
    xs * ys
}

sumProd(rnorm(10), letters[1:10])
\end{lstlisting}

\begin{verbatim}

Error in x^2 : argumento no-numérico para operador binario
\end{verbatim}
\end{frame}

\begin{frame}[label={sec:org37c4363},fragile]{Analizar antes de que ocurra: \texttt{trace}}
 \begin{itemize}
\item \texttt{trace} permite mayor control que \texttt{debug}
\end{itemize}
\lstset{language=r,label= ,caption= ,captionpos=b,numbers=none}
\begin{lstlisting}
trace(sumProd, tracer=browser, exit=browser)
\end{lstlisting}

\begin{verbatim}
[1] "sumProd"
\end{verbatim}


\begin{itemize}
\item La función queda modificada
\end{itemize}
\lstset{language=r,label= ,caption= ,captionpos=b,numbers=none}
\begin{lstlisting}
sumProd
\end{lstlisting}

\begin{verbatim}
Object with tracing code, class "functionWithTrace"
Original definition: 
function(x, y, ...){
    xs <- sumSq(x, ...)
    ys <- sumSq(y, ...)
    xs * ys
}

## (to see the tracing code, look at body(object))
\end{verbatim}


\lstset{language=r,label= ,caption= ,captionpos=b,numbers=none}
\begin{lstlisting}
body(sumProd)
\end{lstlisting}

\begin{verbatim}
{
    on.exit(.doTrace(browser(), "on exit"))
    {
        .doTrace(browser(), "on entry")
        {
            xs <- sumSq(x, ...)
            ys <- sumSq(y, ...)
            xs * ys
        }
    }
}
\end{verbatim}
\end{frame}

\begin{frame}[label={sec:org75004e4},fragile]{Analizar antes de que ocurra: \texttt{trace}}
 \begin{itemize}
\item Los comandos \texttt{n} y \texttt{c} cambian respecto a \texttt{debug}:
\begin{itemize}
\item \texttt{c} o intro: avanzar un paso.
\item \texttt{n}: continua hasta el final del contexto actual (por ejemplo,
terminar un bucle).
\end{itemize}
\item Para desactivar
\end{itemize}
\lstset{language=r,label= ,caption= ,captionpos=b,numbers=none}
\begin{lstlisting}
untrace(sumProd)
\end{lstlisting}
\end{frame}

\section{Profiling}
\label{sec:org0dfd80d}
\begin{frame}[label={sec:org4ea8038},fragile]{¿Cuánto tarda mi función? \texttt{system.time}}
 Defino una función que rellena una matriz de 10\textsuperscript{6} filas y \texttt{n} columnas con una distribución normal:
\lstset{language=r,label= ,caption= ,captionpos=b,numbers=none}
\begin{lstlisting}
makeNoise <- function(n){
    sapply(seq_len(n), function(i) rnorm(1e6))
}
\end{lstlisting}

\lstset{language=r,label= ,caption= ,captionpos=b,numbers=none}
\begin{lstlisting}
M <- makeNoise(100)
summary(M)
\end{lstlisting}

\begin{verbatim}

       V1                  V2                  V3           
 Min.   :-4.697129   Min.   :-5.086494   Min.   :-5.144483  
 1st Qu.:-0.673276   1st Qu.:-0.674048   1st Qu.:-0.672202  
 Median :-0.001501   Median :-0.000023   Median : 0.001893  
 Mean   :-0.001615   Mean   :-0.000317   Mean   : 0.002187  
 3rd Qu.: 0.673160   3rd Qu.: 0.672780   3rd Qu.: 0.676579  
 Max.   : 4.468142   Max.   : 4.853063   Max.   : 4.741897  
       V4                  V5                  V6           
 Min.   :-4.983313   Min.   :-4.927291   Min.   :-4.684501  
 1st Qu.:-0.675177   1st Qu.:-0.677671   1st Qu.:-0.674668  
 Median :-0.000469   Median :-0.000469   Median : 0.001684  
 Mean   :-0.000161   Mean   :-0.000120   Mean   : 0.001354  
 3rd Qu.: 0.674433   3rd Qu.: 0.674996   3rd Qu.: 0.677215  
 Max.   : 4.819135   Max.   : 5.013616   Max.   : 4.876185  
       V7                  V8                  V9           
 Min.   :-4.606624   Min.   :-4.759520   Min.   :-4.486057  
 1st Qu.:-0.672966   1st Qu.:-0.674023   1st Qu.:-0.674597  
 Median :-0.001805   Median :-0.001627   Median :-0.000094  
 Mean   :-0.001356   Mean   :-0.000174   Mean   :-0.000506  
 3rd Qu.: 0.671379   3rd Qu.: 0.673357   3rd Qu.: 0.673683  
 Max.   : 4.937610   Max.   : 4.769564   Max.   : 4.998585  
      V10                 V11                 V12           
 Min.   :-4.893114   Min.   :-4.973934   Min.   :-5.441027  
 1st Qu.:-0.673309   1st Qu.:-0.674037   1st Qu.:-0.676575  
 Median :-0.000482   Median : 0.000342   Median :-0.001992  
 Mean   : 0.000629   Mean   :-0.000425   Mean   :-0.000971  
 3rd Qu.: 0.674886   3rd Qu.: 0.673628   3rd Qu.: 0.674397  
 Max.   : 4.952198   Max.   : 4.612258   Max.   : 4.602244  
      V13                 V14                 V15           
 Min.   :-4.822871   Min.   :-4.698819   Min.   :-4.846533  
 1st Qu.:-0.674612   1st Qu.:-0.672195   1st Qu.:-0.672971  
 Median : 0.001903   Median : 0.000851   Median : 0.000757  
 Mean   : 0.000426   Mean   : 0.000447   Mean   : 0.001030  
 3rd Qu.: 0.675208   3rd Qu.: 0.672648   3rd Qu.: 0.676183  
 Max.   : 4.921475   Max.   : 5.053218   Max.   : 4.575553  
      V16                 V17                 V18                 V19          
 Min.   :-5.071355   Min.   :-4.600348   Min.   :-4.814996   Min.   :-5.29565  
 1st Qu.:-0.674516   1st Qu.:-0.678022   1st Qu.:-0.672917   1st Qu.:-0.67538  
 Median :-0.001434   Median :-0.002216   Median : 0.000444   Median : 0.00125  
 Mean   : 0.000043   Mean   :-0.002813   Mean   : 0.001087   Mean   :-0.00031  
 3rd Qu.: 0.674611   3rd Qu.: 0.671441   3rd Qu.: 0.674044   3rd Qu.: 0.67437  
 Max.   : 4.628296   Max.   : 4.937067   Max.   : 5.201685   Max.   : 5.21104  
      V20                 V21                 V22           
 Min.   :-5.176395   Min.   :-5.001858   Min.   :-5.046585  
 1st Qu.:-0.676086   1st Qu.:-0.675955   1st Qu.:-0.675688  
 Median :-0.001354   Median :-0.000662   Median :-0.000789  
 Mean   :-0.001603   Mean   :-0.000766   Mean   :-0.001263  
 3rd Qu.: 0.672796   3rd Qu.: 0.674041   3rd Qu.: 0.672625  
 Max.   : 4.751937   Max.   : 4.819381   Max.   : 4.655208  
      V23                 V24                 V25           
 Min.   :-4.784760   Min.   :-5.201011   Min.   :-4.833053  
 1st Qu.:-0.674445   1st Qu.:-0.675459   1st Qu.:-0.673446  
 Median : 0.000580   Median :-0.001455   Median : 0.000137  
 Mean   :-0.000127   Mean   : 0.000025   Mean   : 0.000352  
 3rd Qu.: 0.672743   3rd Qu.: 0.674726   3rd Qu.: 0.673681  
 Max.   : 4.693870   Max.   : 4.810501   Max.   : 4.719421  
      V26                 V27                 V28           
 Min.   :-4.769456   Min.   :-4.582715   Min.   :-4.715943  
 1st Qu.:-0.673722   1st Qu.:-0.676423   1st Qu.:-0.675448  
 Median : 0.001197   Median :-0.001205   Median :-0.000046  
 Mean   : 0.001213   Mean   :-0.000571   Mean   :-0.000512  
 3rd Qu.: 0.676158   3rd Qu.: 0.674560   3rd Qu.: 0.674872  
 Max.   : 5.495472   Max.   : 4.913162   Max.   : 4.762072  
      V29                 V30                 V31           
 Min.   :-4.915755   Min.   :-4.782626   Min.   :-4.968282  
 1st Qu.:-0.673279   1st Qu.:-0.675127   1st Qu.:-0.675411  
 Median : 0.000618   Median : 0.001140   Median :-0.000641  
 Mean   : 0.000089   Mean   : 0.000972   Mean   :-0.000133  
 3rd Qu.: 0.673559   3rd Qu.: 0.675319   3rd Qu.: 0.674341  
 Max.   : 5.625269   Max.   : 4.660672   Max.   : 4.804794  
      V32                 V33                 V34           
 Min.   :-5.208053   Min.   :-5.005482   Min.   :-4.535712  
 1st Qu.:-0.672772   1st Qu.:-0.671251   1st Qu.:-0.675080  
 Median : 0.000001   Median : 0.001536   Median :-0.000139  
 Mean   : 0.000124   Mean   : 0.001502   Mean   : 0.000297  
 3rd Qu.: 0.675108   3rd Qu.: 0.674508   3rd Qu.: 0.675074  
 Max.   : 4.949054   Max.   : 4.620808   Max.   : 4.631801  
      V35                 V36                 V37           
 Min.   :-4.544964   Min.   :-5.181317   Min.   :-4.907276  
 1st Qu.:-0.676433   1st Qu.:-0.672412   1st Qu.:-0.674910  
 Median :-0.000547   Median : 0.001155   Median :-0.000114  
 Mean   :-0.001777   Mean   : 0.001139   Mean   :-0.000170  
 3rd Qu.: 0.673492   3rd Qu.: 0.676382   3rd Qu.: 0.676013  
 Max.   : 4.883498   Max.   : 4.861152   Max.   : 4.538400  
      V38                 V39                 V40           
 Min.   :-4.570870   Min.   :-5.039802   Min.   :-5.086259  
 1st Qu.:-0.673897   1st Qu.:-0.673343   1st Qu.:-0.675440  
 Median :-0.000591   Median : 0.002170   Median :-0.001194  
 Mean   :-0.000153   Mean   : 0.001088   Mean   :-0.000507  
 3rd Qu.: 0.673553   3rd Qu.: 0.676210   3rd Qu.: 0.675422  
 Max.   : 4.972339   Max.   : 4.832430   Max.   : 5.157213  
      V41                 V42                 V43           
 Min.   :-4.759814   Min.   :-4.997090   Min.   :-4.717749  
 1st Qu.:-0.674044   1st Qu.:-0.675320   1st Qu.:-0.672478  
 Median : 0.000208   Median : 0.000092   Median :-0.001294  
 Mean   : 0.000165   Mean   :-0.000037   Mean   :-0.000594  
 3rd Qu.: 0.676443   3rd Qu.: 0.673694   3rd Qu.: 0.672348  
 Max.   : 4.639199   Max.   : 4.489568   Max.   : 4.839490  
      V44                 V45                 V46           
 Min.   :-4.958688   Min.   :-4.879315   Min.   :-4.704138  
 1st Qu.:-0.673216   1st Qu.:-0.671828   1st Qu.:-0.672962  
 Median : 0.001013   Median :-0.000072   Median : 0.000806  
 Mean   : 0.001118   Mean   : 0.001269   Mean   : 0.000716  
 3rd Qu.: 0.673817   3rd Qu.: 0.675338   3rd Qu.: 0.676459  
 Max.   : 4.756538   Max.   : 4.820473   Max.   : 4.952218  
      V47                 V48                 V49           
 Min.   :-4.842215   Min.   :-4.849413   Min.   :-5.254544  
 1st Qu.:-0.671463   1st Qu.:-0.671403   1st Qu.:-0.676353  
 Median :-0.000034   Median : 0.002342   Median :-0.000442  
 Mean   :-0.000291   Mean   : 0.000888   Mean   :-0.000469  
 3rd Qu.: 0.672152   3rd Qu.: 0.673961   3rd Qu.: 0.674543  
 Max.   : 4.927199   Max.   : 4.942227   Max.   : 4.584801  
      V50                 V51                 V52           
 Min.   :-5.410137   Min.   :-4.839286   Min.   :-4.632591  
 1st Qu.:-0.674206   1st Qu.:-0.674121   1st Qu.:-0.675837  
 Median : 0.001548   Median : 0.001679   Median : 0.000597  
 Mean   : 0.000115   Mean   : 0.000544   Mean   :-0.000806  
 3rd Qu.: 0.674398   3rd Qu.: 0.675121   3rd Qu.: 0.674094  
 Max.   : 4.873526   Max.   : 4.696721   Max.   : 4.879410  
      V53                 V54                 V55           
 Min.   :-4.869369   Min.   :-4.828177   Min.   :-4.792432  
 1st Qu.:-0.675837   1st Qu.:-0.674225   1st Qu.:-0.673689  
 Median :-0.001148   Median : 0.001099   Median : 0.000635  
 Mean   :-0.001131   Mean   : 0.000781   Mean   : 0.000476  
 3rd Qu.: 0.673272   3rd Qu.: 0.674586   3rd Qu.: 0.674144  
 Max.   : 5.404056   Max.   : 4.849851   Max.   : 4.642590  
      V56                 V57                 V58           
 Min.   :-4.420110   Min.   :-5.210427   Min.   :-5.115755  
 1st Qu.:-0.675826   1st Qu.:-0.673421   1st Qu.:-0.674553  
 Median : 0.000332   Median : 0.000542   Median :-0.001201  
 Mean   :-0.000487   Mean   : 0.000353   Mean   :-0.000137  
 3rd Qu.: 0.674770   3rd Qu.: 0.674478   3rd Qu.: 0.673922  
 Max.   : 4.802298   Max.   : 4.841936   Max.   : 4.894869  
      V59                 V60                 V61           
 Min.   :-4.736686   Min.   :-4.987290   Min.   :-4.748928  
 1st Qu.:-0.673231   1st Qu.:-0.673690   1st Qu.:-0.672966  
 Median : 0.000609   Median : 0.000908   Median :-0.000005  
 Mean   : 0.000440   Mean   : 0.000978   Mean   : 0.000596  
 3rd Qu.: 0.674495   3rd Qu.: 0.675084   3rd Qu.: 0.674745  
 Max.   : 4.974113   Max.   : 4.590604   Max.   : 4.815448  
      V62                 V63                 V64           
 Min.   :-4.560318   Min.   :-4.951513   Min.   :-4.920279  
 1st Qu.:-0.675278   1st Qu.:-0.675598   1st Qu.:-0.672393  
 Median :-0.002135   Median :-0.000433   Median : 0.000778  
 Mean   :-0.000972   Mean   :-0.000941   Mean   : 0.000278  
 3rd Qu.: 0.674811   3rd Qu.: 0.674748   3rd Qu.: 0.674196  
 Max.   : 4.604551   Max.   : 4.705857   Max.   : 4.933980  
      V65                 V66                 V67           
 Min.   :-4.732589   Min.   :-4.766184   Min.   :-5.000064  
 1st Qu.:-0.673558   1st Qu.:-0.674616   1st Qu.:-0.674405  
 Median :-0.001908   Median : 0.000456   Median : 0.001280  
 Mean   : 0.000088   Mean   : 0.000204   Mean   : 0.000958  
 3rd Qu.: 0.673893   3rd Qu.: 0.675196   3rd Qu.: 0.675694  
 Max.   : 4.673648   Max.   : 4.593069   Max.   : 5.194548  
      V68                 V69                 V70           
 Min.   :-4.924575   Min.   :-4.798559   Min.   :-5.372333  
 1st Qu.:-0.673609   1st Qu.:-0.674562   1st Qu.:-0.672871  
 Median : 0.001479   Median : 0.000608   Median : 0.001651  
 Mean   : 0.000681   Mean   :-0.000374   Mean   : 0.000679  
 3rd Qu.: 0.674296   3rd Qu.: 0.675299   3rd Qu.: 0.675174  
 Max.   : 5.296182   Max.   : 4.756053   Max.   : 4.873048  
      V71                 V72                 V73           
 Min.   :-4.637978   Min.   :-4.832514   Min.   :-4.595978  
 1st Qu.:-0.675686   1st Qu.:-0.675648   1st Qu.:-0.674533  
 Median :-0.001953   Median : 0.000064   Median :-0.001275  
 Mean   :-0.001555   Mean   :-0.000840   Mean   :-0.000395  
 3rd Qu.: 0.671865   3rd Qu.: 0.674422   3rd Qu.: 0.672037  
 Max.   : 4.633471   Max.   : 4.628628   Max.   : 5.137100  
      V74                 V75                 V76           
 Min.   :-4.783647   Min.   :-4.777854   Min.   :-5.083633  
 1st Qu.:-0.674341   1st Qu.:-0.676715   1st Qu.:-0.674155  
 Median :-0.001364   Median : 0.000192   Median :-0.000379  
 Mean   :-0.000480   Mean   :-0.000253   Mean   : 0.000402  
 3rd Qu.: 0.674250   3rd Qu.: 0.676088   3rd Qu.: 0.675277  
 Max.   : 5.173989   Max.   : 4.800704   Max.   : 4.920299  
      V77                 V78                 V79           
 Min.   :-4.929574   Min.   :-4.786791   Min.   :-4.740853  
 1st Qu.:-0.676746   1st Qu.:-0.673485   1st Qu.:-0.676100  
 Median :-0.000961   Median : 0.001988   Median :-0.001468  
 Mean   :-0.001248   Mean   : 0.002005   Mean   :-0.000874  
 3rd Qu.: 0.673981   3rd Qu.: 0.676627   3rd Qu.: 0.673314  
 Max.   : 4.879963   Max.   : 5.358316   Max.   : 4.715124  
      V80                 V81                 V82           
 Min.   :-5.075354   Min.   :-4.913480   Min.   :-4.954405  
 1st Qu.:-0.673489   1st Qu.:-0.675919   1st Qu.:-0.674136  
 Median : 0.001091   Median : 0.000948   Median : 0.001028  
 Mean   : 0.001123   Mean   : 0.000065   Mean   : 0.000389  
 3rd Qu.: 0.674574   3rd Qu.: 0.676413   3rd Qu.: 0.675357  
 Max.   : 4.757778   Max.   : 4.718851   Max.   : 4.651396  
      V83                 V84                 V85           
 Min.   :-4.603997   Min.   :-4.913549   Min.   :-4.979754  
 1st Qu.:-0.676505   1st Qu.:-0.673239   1st Qu.:-0.675702  
 Median :-0.002669   Median : 0.001292   Median :-0.001124  
 Mean   :-0.002476   Mean   : 0.001252   Mean   :-0.000474  
 3rd Qu.: 0.674596   3rd Qu.: 0.675289   3rd Qu.: 0.673350  
 Max.   : 5.251794   Max.   : 4.890851   Max.   : 4.584618  
      V86                 V87                 V88           
 Min.   :-4.935014   Min.   :-5.226603   Min.   :-4.991085  
 1st Qu.:-0.671238   1st Qu.:-0.674180   1st Qu.:-0.673474  
 Median : 0.000475   Median : 0.001644   Median : 0.000496  
 Mean   : 0.001124   Mean   : 0.000198   Mean   : 0.000177  
 3rd Qu.: 0.674834   3rd Qu.: 0.675379   3rd Qu.: 0.673509  
 Max.   : 5.561107   Max.   : 4.641129   Max.   : 5.404080  
      V89                 V90                 V91           
 Min.   :-4.958053   Min.   :-5.250973   Min.   :-4.720112  
 1st Qu.:-0.672739   1st Qu.:-0.674843   1st Qu.:-0.675115  
 Median :-0.000846   Median :-0.000256   Median :-0.002661  
 Mean   : 0.000108   Mean   : 0.000441   Mean   :-0.001701  
 3rd Qu.: 0.673058   3rd Qu.: 0.676608   3rd Qu.: 0.672111  
 Max.   : 4.593092   Max.   : 4.591916   Max.   : 4.582245  
      V92                 V93                 V94           
 Min.   :-4.983946   Min.   :-4.647100   Min.   :-4.304167  
 1st Qu.:-0.671601   1st Qu.:-0.671354   1st Qu.:-0.672783  
 Median :-0.000571   Median : 0.002618   Median : 0.000019  
 Mean   : 0.000342   Mean   : 0.001548   Mean   : 0.000714  
 3rd Qu.: 0.673882   3rd Qu.: 0.675053   3rd Qu.: 0.674838  
 Max.   : 4.651265   Max.   : 4.738345   Max.   : 5.216128  
      V95                 V96                 V97           
 Min.   :-4.915043   Min.   :-5.008904   Min.   :-4.742168  
 1st Qu.:-0.674146   1st Qu.:-0.675606   1st Qu.:-0.674595  
 Median :-0.000521   Median : 0.000867   Median :-0.001365  
 Mean   :-0.000703   Mean   : 0.000215   Mean   : 0.000236  
 3rd Qu.: 0.674252   3rd Qu.: 0.676486   3rd Qu.: 0.676038  
 Max.   : 4.769609   Max.   : 4.959832   Max.   : 4.869743  
      V98                 V99                 V100          
 Min.   :-4.855720   Min.   :-4.692284   Min.   :-5.059369  
 1st Qu.:-0.675190   1st Qu.:-0.674815   1st Qu.:-0.674610  
 Median :-0.000847   Median : 0.000043   Median : 0.000172  
 Mean   :-0.000388   Mean   : 0.000349   Mean   :-0.000193  
 3rd Qu.: 0.674963   3rd Qu.: 0.675579   3rd Qu.: 0.674563  
 Max.   : 4.812510   Max.   : 4.478586   Max.   : 4.750664
\end{verbatim}
\end{frame}

\begin{frame}[label={sec:org4f3d3e4},fragile]{Diferentes formas de sumar}
 \texttt{system.time} mide el tiempo de CPU que consume un código\footnote{Para entender la diferencia entre \texttt{user} y \texttt{system} véase explicación \href{http://r.789695.n4.nabble.com/Meaning-of-proc-time-tp2303263p2306691.html}{aquí}.}.

\lstset{language=r,label= ,caption= ,captionpos=b,numbers=none}
\begin{lstlisting}
system.time({
    suma1 <- numeric(1e6)
    for(i in 1:1e6) suma1[i] <- sum(M[i,])
})
\end{lstlisting}

\begin{verbatim}

 user  system elapsed 
1.738   0.008   2.445
\end{verbatim}


\lstset{language=r,label= ,caption= ,captionpos=b,numbers=none}
\begin{lstlisting}
system.time(suma2 <- apply(M, 1, sum))
\end{lstlisting}

\begin{verbatim}
 user  system elapsed 
2.426   0.047   2.481
\end{verbatim}


\lstset{language=r,label= ,caption= ,captionpos=b,numbers=none}
\begin{lstlisting}
system.time(suma3 <- rowSums(M))
\end{lstlisting}

\begin{verbatim}
 user  system elapsed 
0.299   0.000   0.300
\end{verbatim}
\end{frame}


\begin{frame}[label={sec:org1582a3d},fragile]{¿Cuánto tarda cada parte de mi función?: \texttt{Rprof}}
 \begin{itemize}
\item Usaremos un fichero temporal
\end{itemize}
\lstset{language=r,label= ,caption= ,captionpos=b,numbers=none}
\begin{lstlisting}
tmp <- tempfile()
\end{lstlisting}

\begin{itemize}
\item Activamos la toma de información
\end{itemize}
\lstset{language=r,label= ,caption= ,captionpos=b,numbers=none}
\begin{lstlisting}
Rprof(tmp)
\end{lstlisting}

\begin{itemize}
\item Ejecutamos el código a analizar
\end{itemize}
\lstset{language=r,label= ,caption= ,captionpos=b,numbers=none}
\begin{lstlisting}
suma1 <- numeric(1e6)
for(i in 1:1e6) suma1[i] <- sum(M[i,])

suma2 <- apply(M, 1, FUN = sum)

suma3 <- rowSums(M)
\end{lstlisting}
\end{frame}

\begin{frame}[label={sec:org4222e4c},fragile]{¿Cuánto tarda cada parte de mi función?: \texttt{Rprof}}
 \begin{itemize}
\item Paramos el análisis
\end{itemize}
\lstset{language=r,label= ,caption= ,captionpos=b,numbers=none}
\begin{lstlisting}
Rprof()
\end{lstlisting}

\begin{itemize}
\item Extraemos el resumen
\end{itemize}
\lstset{language=r,label= ,caption= ,captionpos=b,numbers=none}
\begin{lstlisting}
summaryRprof(tmp)
\end{lstlisting}

\begin{verbatim}
$by.self
                self.time self.pct total.time total.pct
"apply"              1.78    50.28       2.86     80.79
"aperm.default"      0.56    15.82       0.56     15.82
"FUN"                0.44    12.43       0.44     12.43
"rowSums"            0.40    11.30       0.40     11.30
"sum"                0.28     7.91       0.28      7.91
"lengths"            0.04     1.13       0.04      1.13
"unlist"             0.04     1.13       0.04      1.13

$by.total
                total.time total.pct self.time self.pct
"apply"               2.86     80.79      1.78    50.28
"aperm.default"       0.56     15.82      0.56    15.82
"aperm"               0.56     15.82      0.00     0.00
"FUN"                 0.44     12.43      0.44    12.43
"rowSums"             0.40     11.30      0.40    11.30
"sum"                 0.28      7.91      0.28     7.91
"lengths"             0.04      1.13      0.04     1.13
"unlist"              0.04      1.13      0.04     1.13

$sample.interval
[1] 0.02

$sampling.time
[1] 3.54
\end{verbatim}
\end{frame}

\section{Ejercicios}
\label{sec:org3292dd1}
\begin{frame}[label={sec:orgccd7ab2}]{Áreas de figuras geométricas}
Escribe una función que calcule el área de un círculo, un triángulo o un cuadrado. La función empleará, a su vez, una función diferente definida para cada caso.
\end{frame}

\begin{frame}[label={sec:org882d42e},fragile]{Conversión de temperaturas}
   Escribe una función para realizar la conversión de temperaturas. La función trabajará a partir de un valor (número real) y una letra. La letra indica la escala en la que se introduce esa temperatura. Si la letra es 'C', la temperatura se convertirá de grados centígrados a Fahrenheit. Si la letra es 'F' la temperatura se convertirá de grados Fahrenheit a grados Centígrados. 
Se usarán 2 funciones auxiliares, \texttt{cent2fahr} y \texttt{fahr2cent} para convertir de una escala a otra. Estas funciones aceptan un parámetro (la temperatura en una escala) y devuelven el valor en la otra escala. 

Nota: La relación entre ambas escalas es \(T_F = 9/5 \cdot T_C + 32\)
\end{frame}

\begin{frame}[label={sec:orgf5cc504}]{Tablas de multiplicar}
Construye un programa que muestre por pantalla las tablas de multiplicar del 1 al 10, a partir de dos funciones específicas. La primera función debe devolver el producto de dos valores numéricos enteros dados como parámetros. La segunda función debe mostrar por pantalla la tabla de multiplicar de un número dado como parámetro.
\end{frame}

\begin{frame}[label={sec:orga4915e2},fragile]{Números combinatorios}
 Escribe una función que calcule y muestre en pantalla el número combinatorio a partir de los valores \texttt{n} y \texttt{k}.

\[
 nk = \frac{n!}{(n - k)! \cdot k!}
 \]

Esta función debe estar construida en base a dos funciones auxiliares, una para calcular el factorial de un número, y otra para calcular el número combinatorio.
\end{frame}

\begin{frame}[label={sec:orgb0edaa6},fragile]{Fibonacci}
 Escribe una \alert{función recursiva} que genere los \texttt{n} primeros términos de la serie de Fibonacci. Esta función aceptará el número entero \texttt{n} como argumento. Este valor debe ser positivo, de forma que si el usuario introduce un valor negativo la función devolverá un error.

Nota: En la serie de Fibonacci los dos primeros números son 1, y el resto se obtiene sumando los dos anteriores: \(1, 1, 2, 3, 5, 8, 13, 21, \ldots\)
\end{frame}

\begin{frame}[label={sec:org7fc2398},fragile]{Serie de Taylor}
 Escribe un conjunto de funciones para calcular la aproximación de \(e ^ {-x}\) mediante el desarrollo de Taylor:

\[
  e^{-x} = 1 + \sum_{i = 1}^\infty \frac{(-x)^n}{n!}
  \]

La función principal acepta como argumentos el valor del número real \texttt{x} y el número de términos deseados. Se basará en otras tres funciones: 

\begin{itemize}
\item \texttt{factorial} calcula el factorial de un número entero \texttt{n}.

\item \texttt{potencia} calcula la potencia \texttt{n} de un número real \texttt{x}.

\item \texttt{exponencial} calcula la aproximación anterior de un número real \texttt{x} usando \texttt{n} términos de la serie de Taylor.
\end{itemize}
\end{frame}
\end{document}