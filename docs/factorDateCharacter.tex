% Created 2018-04-09 lun 18:37
% Intended LaTeX compiler: pdflatex
\documentclass[xcolor={usenames,svgnames,dvipsnames}]{beamer}
\usepackage[utf8]{inputenc}
\usepackage[T1]{fontenc}
\usepackage{graphicx}
\usepackage{grffile}
\usepackage{longtable}
\usepackage{wrapfig}
\usepackage{rotating}
\usepackage[normalem]{ulem}
\usepackage{amsmath}
\usepackage{textcomp}
\usepackage{amssymb}
\usepackage{capt-of}
\usepackage{hyperref}
\usepackage{color}
\usepackage{listings}
\usepackage[spanish]{babel}
\usecolortheme{rose}
\setbeamercolor{alerted text}{fg=Blue}
\setbeamerfont{alerted text}{series=\bfseries}
\setbeamercolor{block title}{bg=structure.fg!20!bg!50!bg}
\setbeamercolor{block body}{use=block title,bg=block title.bg}
\setbeamertemplate{navigation symbols}{}
\AtBeginSubsection[]{\begin{frame}[plain]\tableofcontents[currentsubsection,sectionstyle=show/shaded,subsectionstyle=show/shaded/hide]\end{frame}}
\lstset{keywordstyle=\color{blue}, commentstyle=\color{gray!90}, basicstyle=\ttfamily\small, columns=fullflexible, breaklines=true,linewidth=\textwidth, backgroundcolor=\color{gray!23}, basewidth={0.5em,0.4em}, literate={á}{{\'a}}1 {ñ}{{\~n}}1 {é}{{\'e}}1 {ó}{{\'o}}1 {º}{{\textordmasculine}}1, showstringspaces=false}
\usepackage{mathpazo}
\hypersetup{colorlinks=true, linkcolor=Blue, urlcolor=Blue}
\usepackage{fancyvrb}
\DefineVerbatimEnvironment{verbatim}{Verbatim}{fontsize=\tiny, formatcom = {\color{black!70}}}
\beamertemplatenavigationsymbolsempty
\setbeamertemplate{footline}[frame number]
\AtBeginSection[]{\begin{frame}[plain]\tableofcontents[currentsection,sectionstyle=show/shaded]\end{frame}}
\usetheme{Goettingen}
\usefonttheme{serif}
\author{Oscar Perpiñán Lamigueiro \\ \url{http://oscarperpinan.github.io}}
\date{}
\title{Factores, fechas y caracteres}
\hypersetup{
 pdfauthor={Oscar Perpiñán Lamigueiro \\ \url{http://oscarperpinan.github.io}},
 pdftitle={Factores, fechas y caracteres},
 pdfkeywords={},
 pdfsubject={},
 pdfcreator={Emacs 25.2.2 (Org mode 9.1.9)}, 
 pdflang={Spanish}}
\begin{document}

\maketitle


\section{\texttt{factor}}
\label{sec:org9843d92}
\begin{frame}[fragile,label={sec:orgcd2c6eb}]{Construimos un ejemplo}
 \lstset{language=r,label= ,caption= ,captionpos=b,numbers=none}
\begin{lstlisting}
  N <- 100
  edad <- sample(seq(18, 40, 1), N, replace=TRUE)
  summary(edad)
\end{lstlisting}

\begin{verbatim}
 Min. 1st Qu.  Median    Mean 3rd Qu.    Max. 
18.00   25.00   29.00   29.36   35.00   40.00
\end{verbatim}

\lstset{language=r,label= ,caption= ,captionpos=b,numbers=none}
\begin{lstlisting}
  sexo <- sample(c('H', 'M'), N, replace=TRUE)
  class(sexo)
  summary(sexo)
\end{lstlisting}

\begin{verbatim}
[1] "character"
   Length     Class      Mode 
      100 character character
\end{verbatim}
\end{frame}

\begin{frame}[fragile,label={sec:org6c4d0a8}]{Una variable cualitativa se define con \texttt{factor}}
 \lstset{language=r,label= ,caption= ,captionpos=b,numbers=none}
\begin{lstlisting}
  sexo <- factor(sexo)
\end{lstlisting}

\lstset{language=r,label= ,caption= ,captionpos=b,numbers=none}
\begin{lstlisting}
  class(sexo)
\end{lstlisting}

\begin{verbatim}
[1] "factor"
\end{verbatim}

\lstset{language=r,label= ,caption= ,captionpos=b,numbers=none}
\begin{lstlisting}
  summary(sexo)
\end{lstlisting}

\begin{verbatim}
 H  M 
56 44
\end{verbatim}

\lstset{language=r,label= ,caption= ,captionpos=b,numbers=none}
\begin{lstlisting}
  levels(sexo)
\end{lstlisting}

\begin{verbatim}
[1] "H" "M"
\end{verbatim}

\lstset{language=r,label= ,caption= ,captionpos=b,numbers=none}
\begin{lstlisting}
  nlevels(sexo)
\end{lstlisting}

\begin{verbatim}
[1] 2
\end{verbatim}
\end{frame}

\begin{frame}[fragile,label={sec:org08bd5e7}]{Los \texttt{factor} sirven para agrupar}
 \begin{itemize}
\item Con la función \texttt{table}
\end{itemize}
\lstset{language=r,label= ,caption= ,captionpos=b,numbers=none}
\begin{lstlisting}
  table(edad > 30, sexo)
\end{lstlisting}

\begin{verbatim}
     sexo
       H  M
FALSE 33 26
TRUE  23 18
\end{verbatim}

\lstset{language=r,label= ,caption= ,captionpos=b,numbers=none}
\begin{lstlisting}
  table(edad %in% 20:30, sexo)
\end{lstlisting}

\begin{verbatim}
     sexo
       H  M
FALSE 28 22
TRUE  28 22
\end{verbatim}
\end{frame}

\begin{frame}[fragile,label={sec:org3d660f4}]{Los \texttt{factor} sirven para agrupar}
 \begin{itemize}
\item Con \texttt{tapply} o \texttt{aggregate}
\end{itemize}
\lstset{language=r,label= ,caption= ,captionpos=b,numbers=none}
\begin{lstlisting}
tapply(edad, sexo, mean)
\end{lstlisting}

\begin{verbatim}
       H        M 
29.28571 29.45455
\end{verbatim}

\lstset{language=r,label= ,caption= ,captionpos=b,numbers=none}
\begin{lstlisting}
aggregate(edad ~ sexo, FUN=median)
\end{lstlisting}

\begin{verbatim}
  sexo edad
1    H   29
2    M   30
\end{verbatim}
\end{frame}

\begin{frame}[fragile,label={sec:orgda3c5ae}]{Los factores sirven para separar}
 \lstset{language=r,label= ,caption= ,captionpos=b,numbers=none}
\begin{lstlisting}
  edadSexo <- split(edad, sexo)
  class(edadSexo)
\end{lstlisting}

\begin{verbatim}
[1] "list"
\end{verbatim}

\lstset{language=r,label= ,caption= ,captionpos=b,numbers=none}
\begin{lstlisting}
  sapply(edadSexo, mean)
\end{lstlisting}

\begin{verbatim}
       H        M 
29.28571 29.45455
\end{verbatim}
\end{frame}

\begin{frame}[fragile,label={sec:org9e98f84}]{Los \texttt{factor} se pueden generar a partir de variables numéricas}
 \begin{itemize}
\item Por ejemplo, con \texttt{cut}
\end{itemize}
\lstset{language=r,label= ,caption= ,captionpos=b,numbers=none}
\begin{lstlisting}
  gEdad <- cut(edad, breaks=4)
  class(gEdad)
\end{lstlisting}

\begin{verbatim}
[1] "factor"
\end{verbatim}

\lstset{language=r,label= ,caption= ,captionpos=b,numbers=none}
\begin{lstlisting}
  levels(gEdad)
\end{lstlisting}

\begin{verbatim}
[1] "(18,23.5]" "(23.5,29]" "(29,34.5]" "(34.5,40]"
\end{verbatim}

\begin{itemize}
\item Nuevamente \texttt{table}
\end{itemize}
\lstset{language=r,label= ,caption= ,captionpos=b,numbers=none}
\begin{lstlisting}
  table(gEdad)
\end{lstlisting}

\begin{verbatim}
gEdad
(18,23.5] (23.5,29] (29,34.5] (34.5,40] 
       18        34        22        26
\end{verbatim}

\lstset{language=r,label= ,caption= ,captionpos=b,numbers=none}
\begin{lstlisting}
  table(gEdad, sexo)
\end{lstlisting}

\begin{verbatim}
           sexo
gEdad        H  M
  (18,23.5]  9  9
  (23.5,29] 22 12
  (29,34.5] 11 11
  (34.5,40] 14 12
\end{verbatim}
\end{frame}

\section{Fechas}
\label{sec:org587eff4}
\begin{frame}[fragile,label={sec:orga0fa92c}]{\texttt{Date}}
 \lstset{language=r,label= ,caption= ,captionpos=b,numbers=none}
\begin{lstlisting}
  as.Date('2013-02-06')
\end{lstlisting}

\begin{verbatim}
[1] "2013-02-06"
\end{verbatim}

\lstset{language=r,label= ,caption= ,captionpos=b,numbers=none}
\begin{lstlisting}
  as.Date('2013/02/06')
\end{lstlisting}

\begin{verbatim}
[1] "2013-02-06"
\end{verbatim}

\lstset{language=r,label= ,caption= ,captionpos=b,numbers=none}
\begin{lstlisting}
  as.Date('06.02.2013')
\end{lstlisting}

\begin{verbatim}
Error in charToDate(x) : 
  character string is not in a standard unambiguous format
\end{verbatim}

\lstset{language=r,label= ,caption= ,captionpos=b,numbers=none}
\begin{lstlisting}
  as.Date('06.02.2013', format='%d.%m.%Y')
\end{lstlisting}

\begin{verbatim}
[1] "2013-02-06"
\end{verbatim}

\lstset{language=r,label= ,caption= ,captionpos=b,numbers=none}
\begin{lstlisting}
  as.Date(37, origin='2013-01-01')
\end{lstlisting}

\begin{verbatim}
[1] "2013-02-07"
\end{verbatim}
\end{frame}

\begin{frame}[fragile,label={sec:org4ac0e7f}]{Secuencias temporales con \texttt{Date}}
 \lstset{language=r,label= ,caption= ,captionpos=b,numbers=none}
\begin{lstlisting}
  seq(as.Date('2004-01-01'), by='day', length=10)
\end{lstlisting}

\begin{verbatim}
[1] "2004-01-01" "2004-01-02" "2004-01-03" "2004-01-04" "2004-01-05"
[6] "2004-01-06" "2004-01-07" "2004-01-08" "2004-01-09" "2004-01-10"
\end{verbatim}

\lstset{language=r,label= ,caption= ,captionpos=b,numbers=none}
\begin{lstlisting}
  seq(as.Date('2004-01-01'), by='month', length=10)
\end{lstlisting}

\begin{verbatim}
[1] "2004-01-01" "2004-02-01" "2004-03-01" "2004-04-01" "2004-05-01"
[6] "2004-06-01" "2004-07-01" "2004-08-01" "2004-09-01" "2004-10-01"
\end{verbatim}

\lstset{language=r,label= ,caption= ,captionpos=b,numbers=none}
\begin{lstlisting}
  seq(as.Date('2004-01-01'), by='10 day', length=10)
\end{lstlisting}

\begin{verbatim}
[1] "2004-01-01" "2004-01-11" "2004-01-21" "2004-01-31" "2004-02-10"
[6] "2004-02-20" "2004-03-01" "2004-03-11" "2004-03-21" "2004-03-31"
\end{verbatim}
\end{frame}

\begin{frame}[fragile,label={sec:org9ac4811}]{POSIXct}
 \begin{itemize}
\item \texttt{help(format.POSIXct)}
\end{itemize}
\lstset{language=r,label= ,caption= ,captionpos=b,numbers=none}
\begin{lstlisting}
  as.POSIXct('2013-02-06')
\end{lstlisting}

\begin{verbatim}
[1] "2013-02-06 CET"
\end{verbatim}

\lstset{language=r,label= ,caption= ,captionpos=b,numbers=none}
\begin{lstlisting}
  ISOdate(2013, 2, 7)
\end{lstlisting}

\begin{verbatim}
[1] "2013-02-07 12:00:00 GMT"
\end{verbatim}

\lstset{language=r,label= ,caption= ,captionpos=b,numbers=none}
\begin{lstlisting}
hoy <- as.POSIXct('2013-02-06')

format(hoy, '%Y')
\end{lstlisting}

\begin{verbatim}
[1] "2013"
\end{verbatim}

\lstset{language=r,label= ,caption= ,captionpos=b,numbers=none}
\begin{lstlisting}
format(hoy, '%d')
\end{lstlisting}

\begin{verbatim}
[1] "06"
\end{verbatim}

\lstset{language=r,label= ,caption= ,captionpos=b,numbers=none}
\begin{lstlisting}
format(hoy, '%m')
\end{lstlisting}

\begin{verbatim}
[1] "02"
\end{verbatim}

\lstset{language=r,label= ,caption= ,captionpos=b,numbers=none}
\begin{lstlisting}
format(hoy, '%b')
\end{lstlisting}

\begin{verbatim}
[1] "feb"
\end{verbatim}

\lstset{language=r,label= ,caption= ,captionpos=b,numbers=none}
\begin{lstlisting}
format(hoy, '%d de %B de %Y')
\end{lstlisting}

\begin{verbatim}
[1] "06 de febrero de 2013"
\end{verbatim}
\end{frame}

\begin{frame}[fragile,label={sec:org580513d}]{\texttt{POSIxct}}
 \lstset{language=r,label= ,caption= ,captionpos=b,numbers=none}
\begin{lstlisting}
  hora <- Sys.time()
  hora
\end{lstlisting}

\begin{verbatim}
[1] "2018-04-09 18:37:09 CEST"
\end{verbatim}

\lstset{language=r,label= ,caption= ,captionpos=b,numbers=none}
\begin{lstlisting}
  format(hora, '%H:%M:%S')
\end{lstlisting}

\begin{verbatim}
[1] "18:37:09"
\end{verbatim}

\lstset{language=r,label= ,caption= ,captionpos=b,numbers=none}
\begin{lstlisting}
  format(hora, '%H horas, %M minutos y %S segundos')
\end{lstlisting}

\begin{verbatim}
[1] "18 horas, 37 minutos y 09 segundos"
\end{verbatim}
\end{frame}

\begin{frame}[fragile,label={sec:orgb7bdebe}]{Secuencias temporales con \texttt{POSIXct}}
 \lstset{language=r,label= ,caption= ,captionpos=b,numbers=none}
\begin{lstlisting}
seq(as.POSIXct('2004-01-01'), by='month', length=10)
\end{lstlisting}

\begin{verbatim}
[1] "2004-01-01 CET"  "2004-02-01 CET"  "2004-03-01 CET"  "2004-04-01 CEST"
[5] "2004-05-01 CEST" "2004-06-01 CEST" "2004-07-01 CEST" "2004-08-01 CEST"
[9] "2004-09-01 CEST" "2004-10-01 CEST"
\end{verbatim}

\lstset{language=r,label= ,caption= ,captionpos=b,numbers=none}
\begin{lstlisting}
seq(as.POSIXct('2004-01-01 10:00:00'), by='15 min', length=10)
\end{lstlisting}

\begin{verbatim}
[1] "2004-01-01 10:00:00 CET" "2004-01-01 10:15:00 CET"
[3] "2004-01-01 10:30:00 CET" "2004-01-01 10:45:00 CET"
[5] "2004-01-01 11:00:00 CET" "2004-01-01 11:15:00 CET"
[7] "2004-01-01 11:30:00 CET" "2004-01-01 11:45:00 CET"
[9] "2004-01-01 12:00:00 CET" "2004-01-01 12:15:00 CET"
\end{verbatim}
\end{frame}

\begin{frame}[fragile,label={sec:org93e3f40}]{Zonas horarias}
 \lstset{language=r,label= ,caption= ,captionpos=b,numbers=none}
\begin{lstlisting}
  as.POSIXct('2013-02-06 15:30:00',
             tz='GMT')
\end{lstlisting}

\begin{verbatim}
[1] "2013-02-06 15:30:00 GMT"
\end{verbatim}

\lstset{language=r,label= ,caption= ,captionpos=b,numbers=none}
\begin{lstlisting}
  as.POSIXct('2013-02-06 15:30:00',
             tz='Europe/Madrid')
\end{lstlisting}

\begin{verbatim}
[1] "2013-02-06 15:30:00 CET"
\end{verbatim}

\lstset{language=r,label= ,caption= ,captionpos=b,numbers=none}
\begin{lstlisting}
hawaii <- as.POSIXct('2013-02-06 15:30:00', tz='HST')
## Character
format(hawaii, tz='GMT')
\end{lstlisting}

\begin{verbatim}
[1] "2013-02-07 01:30:00"
\end{verbatim}

\lstset{language=r,label= ,caption= ,captionpos=b,numbers=none}
\begin{lstlisting}
## POSIXct
as.POSIXct(format(hawaii, tz='GMT'), tz='GMT')
\end{lstlisting}

\begin{verbatim}
[1] "2013-02-07 01:30:00 GMT"
\end{verbatim}
\end{frame}

\section{Caracteres}
\label{sec:orgf8229cf}

\begin{frame}[fragile,label={sec:org16918f5}]{Bastan unas simples comillas}
 \lstset{language=r,label= ,caption= ,captionpos=b,numbers=none}
\begin{lstlisting}
  cadena <- "Hola mundo"
  class(cadena)
\end{lstlisting}

\begin{verbatim}
[1] "character"
\end{verbatim}

\lstset{language=r,label= ,caption= ,captionpos=b,numbers=none}
\begin{lstlisting}
  nchar(cadena)
\end{lstlisting}

\begin{verbatim}
[1] 10
\end{verbatim}
\end{frame}

\begin{frame}[fragile,label={sec:org29b672e}]{Un vector de \texttt{character}}
 \lstset{language=r,label= ,caption= ,captionpos=b,numbers=none}
\begin{lstlisting}
  cadenaVec <- c("Hola mundo", "Hello world")
  nchar(cadenaVec)
\end{lstlisting}

\begin{verbatim}
[1] 10 11
\end{verbatim}

\lstset{language=r,label= ,caption= ,captionpos=b,numbers=none}
\begin{lstlisting}
  length(cadenaVec)
\end{lstlisting}

\begin{verbatim}
[1] 2
\end{verbatim}

\lstset{language=r,label= ,caption= ,captionpos=b,numbers=none}
\begin{lstlisting}
cadenaVec[1]
\end{lstlisting}

\begin{verbatim}
[1] "Hola mundo"
\end{verbatim}
\end{frame}

\begin{frame}[fragile,label={sec:org5bf6808}]{Para mostrarlos usamos \texttt{cat} o \texttt{print}}
 \lstset{language=r,label= ,caption= ,captionpos=b,numbers=none}
\begin{lstlisting}
  a <- 2
  b <- 3
\end{lstlisting}

\lstset{language=r,label= ,caption= ,captionpos=b,numbers=none}
\begin{lstlisting}
  cat('La suma de', a, 'y', b, 'es', a + b, fill=TRUE)
\end{lstlisting}

\begin{verbatim}
La suma de 2 y 3 es 5
\end{verbatim}

\lstset{language=r,label= ,caption= ,captionpos=b,numbers=none}
\begin{lstlisting}
  cat('La suma de', a, 'y', b, 'es', a + b, '\n',
      'La multiplicación de', a, 'por', b, 'es', a*b, '\n')
\end{lstlisting}

\begin{verbatim}
La suma de 2 y 3 es 5 
 La multiplicación de 2 por 3 es 6
\end{verbatim}
\end{frame}

\begin{frame}[fragile,label={sec:org329f850}]{Los \texttt{character} se pueden unir\ldots{}}
 \lstset{language=r,label= ,caption= ,captionpos=b,numbers=none}
\begin{lstlisting}
  paste('Hello', 'World', sep='_')
\end{lstlisting}

\begin{verbatim}
[1] "Hello_World"
\end{verbatim}

\lstset{language=r,label= ,caption= ,captionpos=b,numbers=none}
\begin{lstlisting}
  paste('X', 1:5, sep='.')
\end{lstlisting}

\begin{verbatim}
[1] "X.1" "X.2" "X.3" "X.4" "X.5"
\end{verbatim}

\lstset{language=r,label= ,caption= ,captionpos=b,numbers=none}
\begin{lstlisting}
  paste(c('A', 'B'), 1:5, sep='.')
\end{lstlisting}

\begin{verbatim}
[1] "A.1" "B.2" "A.3" "B.4" "A.5"
\end{verbatim}

\lstset{language=r,label= ,caption= ,captionpos=b,numbers=none}
\begin{lstlisting}
  paste(c('A', 'B'), 1:5, sep='.', collapse='|')
\end{lstlisting}

\begin{verbatim}
[1] "A.1|B.2|A.3|B.4|A.5"
\end{verbatim}
\end{frame}

\begin{frame}[fragile,label={sec:org90184d1}]{\ldots{} y también se pueden separar\ldots{}}
 \lstset{language=r,label= ,caption= ,captionpos=b,numbers=none}
\begin{lstlisting}
  strsplit(cadenaVec, split=' ')
\end{lstlisting}

\begin{verbatim}
[[1]]
[1] "Hola"  "mundo"

[[2]]
[1] "Hello" "world"
\end{verbatim}

\lstset{language=r,label= ,caption= ,captionpos=b,numbers=none}
\begin{lstlisting}
  strsplit(cadenaVec, split='')
\end{lstlisting}

\begin{verbatim}
[[1]]
 [1] "H" "o" "l" "a" " " "m" "u" "n" "d" "o"

[[2]]
 [1] "H" "e" "l" "l" "o" " " "w" "o" "r" "l" "d"
\end{verbatim}

\lstset{language=r,label= ,caption= ,captionpos=b,numbers=none}
\begin{lstlisting}
  chSep <- strsplit(cadenaVec, split=' ')
  class(chSep)
\end{lstlisting}

\begin{verbatim}
[1] "list"
\end{verbatim}

\lstset{language=r,label= ,caption= ,captionpos=b,numbers=none}
\begin{lstlisting}
  length(chSep)
\end{lstlisting}

\begin{verbatim}
[1] 2
\end{verbatim}

\lstset{language=r,label= ,caption= ,captionpos=b,numbers=none}
\begin{lstlisting}
  sapply(chSep, nchar)
\end{lstlisting}

\begin{verbatim}
     [,1] [,2]
[1,]    4    5
[2,]    5    5
\end{verbatim}
\end{frame}

\begin{frame}[fragile,label={sec:orgbb8d965}]{\ldots{} y, por supuesto, manipular}
 \lstset{language=r,label= ,caption= ,captionpos=b,numbers=none}
\begin{lstlisting}
  sub('o', '0', 'Hola Mundo')
\end{lstlisting}

\begin{verbatim}
[1] "H0la Mundo"
\end{verbatim}

\lstset{language=r,label= ,caption= ,captionpos=b,numbers=none}
\begin{lstlisting}
  gsub('o', '0', 'Hola Mundo')
\end{lstlisting}

\begin{verbatim}
[1] "H0la Mund0"
\end{verbatim}

\lstset{language=r,label= ,caption= ,captionpos=b,numbers=none}
\begin{lstlisting}
  substring(cadena, 1) <- 'HOLA'
  cadena
\end{lstlisting}

\begin{verbatim}
[1] "HOLA mundo"
\end{verbatim}

\lstset{language=r,label= ,caption= ,captionpos=b,numbers=none}
\begin{lstlisting}
  tolower(cadena)
\end{lstlisting}

\begin{verbatim}
[1] "hola mundo"
\end{verbatim}

\lstset{language=r,label= ,caption= ,captionpos=b,numbers=none}
\begin{lstlisting}
  toupper(cadena)
\end{lstlisting}

\begin{verbatim}
[1] "HOLA MUNDO"
\end{verbatim}
\end{frame}
\end{document}